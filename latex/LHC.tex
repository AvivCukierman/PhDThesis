\section{Introduction}
The \textit{Large Hadron Collider}~\cite{CERN-Brochure-2017-002-Eng,Pettersson:291782}, or \textit{LHC}, is a high energy particle accelerator and collider located at \textit{CERN} near Geneva, Switzerland.
The LHC is the largest machine ever built, with a circumference of $27$ km, running $100$ m underground and straddling the border between Switzerland and France.
The purpose of the LHC is to produce the highest energy particle collisions ever, with a design center-of-mass energy of $\sqrt{s}=14$ TeV.
A view of the LHC superimposed on an aerial view of Geneva and the surrounding areas can be seen in Figure~\ref{fig:LHC:aerial}, including the four main LHC experiments: ATLAS~\cite{PERF-2007-01}, CMS~\cite{CMS-TDR-08-001}, LHCb~\cite{Alves:2008zz}, and ALICE~\cite{Aamodt:2008zz}.
It was built at a cost of about $4.5$ billion USD over the course of almost $20$ years~\cite{CERN-Brochure-2017-002-Eng}, in addition making it one of the most expensive scientific experiments ever built.

\begin{figure}[htbp]
  \centering 
  \subfloat[]{\includegraphics[width=0.8\textwidth]{figures/{Aerial-photograph-representing-the-Large-Hadron-Collider-with-the-border-between-France}.png}}
  \caption{An image of the LHC (yellow circle) superimposed on an aerial view of Geneva and the surrounding areas. The white dashed lined indicates the French/Swiss border. The two main CERN sites at Meyrin and Pr{\'e}vessin, and the four main LHC experiments (ATLAS, CMS, LHCb, and ALICE) are indicated. Geneva is at the tip of Lac L{\'e}man, and Mont Blanc can be seen in the background. Figure sourced from~\cite{HiggsBridge}.}
  \label{fig:LHC:aerial}
\end{figure}

The main LHC experiments are built around particle interaction points and are themselves extremely impressive scientific apparatuses.
This Thesis uses data gathered at the ATLAS experiment (Chapter~\ref{ch:ATLAS}), which is the largest particle detector ever built by volume, with a diameter of $25$ m and a length of $44$ m~\cite{atlas-discover-detector}.
The ``rival'' experiment to ATLAS is CMS, which is slightly smaller than ATLAS but is the heaviest particle detector ever, at $14000$ tons~\cite{cms-discover-detector}.
The LHC also represents a remarkable international scientific collaboration, with over $10000$ scientists from at least $40$ countries~\cite{CERN-Brochure-2017-002-Eng} working on one of the experiments or on LHC operations.

This Chapter goes over the design of the LHC (Section~\ref{sec:LHC:LHC}), the conditions of particle collisions (Section~\ref{sec:LHC:luminosity}), and finally the data taking history and future plans for the LHC (Section~\ref{sec:LHC:data}).

\section{Design}
\label{sec:LHC:LHC}
The LHC is the final stage of the CERN accelerator complex, which accelerates protons from rest to TeV-scale energies through a series of accelerators, with a final speed of $>99.99999\%$ of the speed of light.
The entire CERN accelerator complex can be seen in Figure~\ref{fig:LHC:LHC}.

\begin{figure}[htbp]
  \centering 
  \subfloat[]{\includegraphics[width=0.6\textwidth]{figures/{AccComplex0700829}.gif}}
  \caption{The CERN accelerator complex. Figure sourced from~\cite{accelerator-complex}.}
  \label{fig:LHC:LHC}
\end{figure}

Protons start as ordinary ionized hydrogen, which are accelerated in the \textit{Linear Accelerator 2} (Linac 2) to an energy of $50$ MeV~\cite{CERN-Brochure-2017-002-Eng}.
They are then passed through the \textit{Proton Synchrotron Booster} (PS Booster) and the \textit{Proton Synchrotron} (PS) to final energies of $1.4$ GeV and $25$ GeV, respectively.
The final intermediate stage before the LHC is the \textit{Super Proton Synchrotron} (SPS), which accerates the protons an injection energy of $450$ GeV.
The LHC itself then accelerates the protons to an energy of up to $7$ TeV, though operating histories have been less than this - $3.5$ TeV in 2010, $4$ TeV in 2011-2012, and $6.5$ TeV in 2015-2018 data taking (Section~\ref{sec:ATLAS:datataking}).

The LHC is also used to accelerate and collide lead ions, which reach energies of up to $2.56$ TeV.

The particles are accelerated using radiofrequency cavities~\cite{cern-accelerators,humphries_1986} .

%Charged particle in magnetic field
%Octants

\section{Luminosity and Pile-up}
\label{sec:LHC:luminosity}
%Emittance, Louisville's theorem
%Luminosity formula
%Pile-up

\section{Data Taking History}
\label{sec:LHC:data}
%Luminosity
%Future plans
