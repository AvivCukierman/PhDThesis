\section{Introduction}
The \textit{Large Hadron Collider}~\cite{CERN-Brochure-2017-002-Eng,Pettersson:291782,Evans:313675,Evans:2008zzb}, or \textit{LHC}, is a high energy particle accelerator and collider located at \textit{CERN} near Geneva, Switzerland.
The LHC is the largest machine ever built, with a circumference of $27$ km, running $100$ m underground and straddling the border between Switzerland and France.
The purpose of the LHC is to produce the highest energy particle collisions ever, with a design center-of-mass energy of $\sqrt{s}=14$ TeV\footnote{The \textit{center-of-mass energy} of two colliding beams is $\sqrt{s}$, where $s$ is the Mandelstam variable $s = (p_1+p_2)^2$, and $p_1$ and $p_2$ are the four-momenta of the two beams. In the case that the beams collide head-on, with $E_1 = E_2 \equiv E \gg m_1 = m_2$, $\sqrt{s} = 2\sqrt{E_1E_2} = 2E$ (which is the case in the LHC). I.e., the protons in each beam have energy $7$ TeV, and they collide basically head-on, so $\sqrt{s}=14$ TeV.}.
A view of the LHC superimposed on an aerial view of Geneva and the surrounding areas can be seen in Figure~\ref{fig:LHC:aerial}, including the four main LHC experiments: ATLAS~\cite{PERF-2007-01}, CMS~\cite{CMS-TDR-08-001}, LHCb~\cite{Alves:2008zz}, and ALICE~\cite{Aamodt:2008zz}.
It was built at a cost of about $4.5$ billion USD over the course of almost $20$ years~\cite{CERN-Brochure-2017-002-Eng}, in addition making it one of the most expensive scientific experiments ever built.

\begin{figure}[htbp]
  \centering 
  \subfloat[]{\includegraphics[width=0.8\textwidth]{figures/{Aerial-photograph-representing-the-Large-Hadron-Collider-with-the-border-between-France}.png}}
  \caption{An image of the LHC (yellow circle) superimposed on an aerial view of Geneva and the surrounding areas. The white dashed lined indicates the French/Swiss border. The two main CERN sites at Meyrin and Pr{\'e}vessin, and the four main LHC experiments (ATLAS, CMS, LHCb, and ALICE) are indicated. Geneva is at the tip of Lac L{\'e}man, and Mont Blanc can be seen in the background. Figure sourced from~\cite{HiggsBridge}.}
  \label{fig:LHC:aerial}
\end{figure}

The main LHC experiments are built around particle interaction points and are themselves extremely impressive scientific apparatuses.
This Thesis uses data gathered at the ATLAS experiment (Chapter~\ref{ch:ATLAS}), which is the largest particle detector ever built by volume, with a diameter of $25$ m and a length of $44$ m~\cite{atlas-discover-detector}.
The ``rival'' experiment to ATLAS is CMS, which is slightly smaller than ATLAS but is the heaviest particle detector ever, at $14000$ tons~\cite{cms-discover-detector}.
The LHC also represents a remarkable international scientific collaboration, with over $10000$ scientists from at least $40$ countries~\cite{CERN-Brochure-2017-002-Eng} working on one of the experiments or on LHC operations.

This Chapter goes over the design of the LHC (Section~\ref{sec:LHC:LHC}), the conditions of particle collisions (Section~\ref{sec:LHC:luminosity}), and finally the data taking history and future plans for the LHC (Section~\ref{sec:LHC:data}).

\section{Design}
\label{sec:LHC:LHC}
The LHC is the final stage of the CERN accelerator complex, which accelerates protons from rest to TeV-scale energies through a series of accelerators, with a final speed of $>99.99999\%$ of the speed of light.
The entire CERN accelerator complex can be seen in Figure~\ref{fig:LHC:LHC}.

\begin{figure}[htbp]
  \centering 
  \subfloat[]{\includegraphics[width=0.6\textwidth]{figures/{AccComplex0700829}.png}}
  \caption{The CERN accelerator complex. Figure sourced from~\cite{accelerator-complex}.}
  \label{fig:LHC:LHC}
\end{figure}

Protons start as ordinary ionized hydrogen, which are accelerated~\cite{CERN-Brochure-2017-002-Eng} in the \textit{Linear Accelerator 2} (Linac 2) to an energy of $50$ MeV.
They are then passed through the \textit{Proton Synchrotron Booster} (PS Booster) and the \textit{Proton Synchrotron} (PS) to final energies of $1.4$ GeV and $25$ GeV, respectively.
The final intermediate stage before the LHC is the \textit{Super Proton Synchrotron} (SPS), which accelerates the protons an injection energy of $450$ GeV.
The LHC itself then accelerates the protons to an energy of up to $7$ TeV, though operating histories have been less than this - $3.5$ TeV in 2010, $4$ TeV in 2011-2012, and $6.5$ TeV in 2015-2018 data taking (Section~\ref{sec:LHC:data}).

The LHC is also used to accelerate and collide lead ions, which reach energies of up to $2.56$ TeV.

The particles are accelerated using radiofrequency cavities~\cite{cern-accelerators,humphries_1986} which in the LHC operate at $400$ MHz - every tenth wave is filled with a \textit{bunch} of about $10^{11}$ protons, for a bunch rate of $40$ MHz, and $2808$ bunches circulating the LHC per proton beam.
There are two beams circulating in opposite directions in adjacent beam pipes, which come together at a few collision points around the ring, with a \textit{bunch crossing} every $25$ ns; the detectors are built around these collision points.

Accelerating charged particles radiate~\cite{jackson_1999}, so that the accelerators have to constantly input energy in order to maintain constant beam energy.
However, the radiation from charged particles moving in circular motion, called \textit{synchrotron radiation}, goes as $\frac{E^2}{m^{4}}$, so this radiation is strongly suppressed for protons relative to electrons and positrons at the same energy.
As a historical note, the LHC reuses the tunnel built for the \textit{Large Electron-Positron Collider}~\cite{LEP-design,Myers:226776,cern-lep}, or \textit{LEP}, which collided electrons and protons at a maximum center-of-mass energy of $209$ GeV.
However, when increasing the energy of the beams for the LHC, it was decided to switch to proton-proton collisions which suppresses the synchrotron radiation.

The particles are kept along a circular trajectory with a perpendicular magnetic field produced by dipole magnets\footnote{There are also quadrupole, sextupole, etc. magnets to provide second-order effects like stabilization and beam focusing.}.
The radius of curvature $R$, magnetic field $B$, and momentum $p$\footnote{Note that, as the particles are highly relativistic, $p=\gamma mv$. Also, since $m_p\approx 1~\GeV{} \ll ~\mathcal{O}(\TeV{})=$ beam energy, $p\approx E$.} are related~\cite{humphries_1986} via
\begin{align}
  BR = \frac{p}{|q|},
  \label{eqn:LHC:chargedparticle}
\end{align}
where $q$ is the charge (for protons, $q=+1$ elementary charges).
As the protons are accelerated around the ring the magnetic field increases proportionally in order to keep the radius fixed.
A given magnet can only support a magnetic field range of $\mathcal{O}(100)$, which is why the energy increases by a factor of $20-25$ in each stage of the accelerator complex before moving to the next stage with a larger $R$.

In the final stage, in the LHC, $1232$ dipole magnets provide a magnetic field of up to $8.3$ T (there are almost $10000$ total magnets, including those with higher orders)~\cite{CERN-Brochure-2017-002-Eng}.
The magnets are made of superconducting NbTi cables cooled down to $1.9$ K with superfluid helium.

The LHC consumes about $750$ GWh per year~\cite{cern-facts-and-figures}, most of which is for the cryogenics system, and CERN in general consumes about $1300$ GWh per year; in comparison, the entire Geneva canton consumes about $3000$ GWh per year\footnote{CERN is actually connected to the French electrical grid, not the Swiss~\cite{cern-powering}.}.

The LHC is split into eight \textit{octants}, each of which supports its own cryostat.
As can be seen in Figure~\ref{fig:LHC:octants}, each octant has some special feature, like the two injection sites, beam dump, beam cleaning, and of course the four collision points.
\begin{figure}[htbp]
  \centering 
  \subfloat[]{\includegraphics[width=0.6\textwidth]{figures/{LHC_octants}.png}}
  \caption{Schematic of the LHC showing the two circulating beams, the four main experiments, and the eight octants with some of their special features. Figure sourced from~\cite{Evans:2008zzb}.}
  \label{fig:LHC:octants}
\end{figure}
ATLAS, in the center of Octant 1, has the distinction of being located at ``Point 1'', which is also near where (on surface level) the main CERN Meyrin campus is, and the easiest to access from downtown Geneva; CMS on the other hand is at ``Point 5'', at the CERN Pr{\'e}vessin campus in the middle of the French countryside. 

\section{Luminosity and Pile-up}
\label{sec:LHC:luminosity}
The rate of collisions is called the \textit{luminosity} $\mathcal{L}$~\cite{griffiths_particles} (or \textit{instantaneous luminosity}), and has units of inverse area per unit time.
The expected number of events $N$ per unit time of a particular process with cross-section $\sigma$ is
\begin{align}
  \frac{dN}{dt} = \mathcal{L}\sigma.
\end{align}

In particle physics a commonly used unit of area is the \textit{barn}, which is equal to $10^{-24}$ cm$^2$~\cite{si-brochure}.
The LHC has a peak luminosity of about $10^{34}~\text{cm}^{-2}\text{s}^{-1} = 10~\text{nb}^{-1}\text{s}^{-1}$~\cite{Evans:2008zzb}\footnote{Just to be clear, $1$ nb$^{-1}$ = $10^{+9}$ b$^{-1}$, etc.}.
As a point of comparison, the total proton-proton cross section at $13$ TeV is about $10^8$ nb, as can be seen in Figure~\ref{fig:LHC:xs}.
While at the LHC the total proton-proton cross section is measured (e.g. in ATLAS, broken down into total elastic cross section~\cite{Aaboud:2016ijx} and total inelastic cross section~\cite{Aaboud:2016mmw}), one primary physics goal of the LHC is to provide inelastic scattering of the proton constituents (partons) at the electroweak scale ($\mathcal{O}(100)$ GeV), e.g. in particular the production of a Higgs boson.
The cross section for these ``interesting'' processes are at least a factor of $10^{-5}$ less than the total cross section, and many (like production of the Higgs boson) are much rarer; therefore in order to produce enough events to claim discovery of a new particle with statistical significance, increasing luminosity is critical.
\begin{figure}[htbp]
  \centering 
  \subfloat[]{\includegraphics[width=0.5\textwidth]{figures/{crosssections2012_v5}.pdf}}
  \caption{Cross sections for $pp$ interactions as a function of $\sqrt{s}$ center-of-mass energy. Figure sourced from~\cite{StirlingXS}.}
  \label{fig:LHC:xs}
\end{figure}

The luminosity at the LHC is given by the formula
\begin{align}
  \mathcal{L} = \frac{N^2 n f \gamma}{4 \pi \epsilon^* \beta^*}F,
\end{align}
where $N$ is the number of particles per bunch, $n$ the number of bunches per beam, $f$ the revolution frequency, $\gamma$ the relativistic factor, $\epsilon^*$ the \textit{normalized emittance}, $\beta^*$ the \textit{beta function} at the collision point, and $F$ a geometric factor based on the angle of the beam crossing.

The \textit{emittance} $\epsilon$~\cite{wilson_accelerators,Holzer:2203629} is a quantity that measures the area or spread in phase space (position-momentum space $x-p$) taken up by the beam - the locus of points in phase space usually forms an ellipse, and the emittance is the area of this ellipse\footnote{More precisely, the set of points in the beam can be considered to be drawn from a two-dimensional Gaussian distribution in phase space, for which the contours of constant probability density are ellipses. The area is appropriately defined according to some integral over this probability density.}.
An example locus of points in phase space can be seen in Figure~\ref{fig:LHC:emittance}.
According to Liouville's theorem~\cite{liouville1838note,gibbs1884fundamental}, the emittance is an invariant for a beam at constant energy being acted on by external magnetic fields (disregarding synchrotron radiation).
As the beams are accelerated to increase the energy and longitudinal momentum, the emittance shrinks as $\frac{1}{p}$, a phenomenon known as \textit{adiabatic damping}.
The \textit{normalized emittance} $\epsilon^* = \frac{p}{m}\epsilon$ takes into this into account and is an invariant for the beam the whole time it is being accelerated.

\begin{figure}[htbp]
  \centering 
  \subfloat[]{\includegraphics[width=0.5\textwidth]{figures/{emittance}.png}}
  \caption{Example locus of points in a beam in phase space, showing the distributions in $x$ and $p_x$. The red ellipse corresponds to a contour of constant probability density, and the area of the ellipse is (up to a constant factor) the emittance $\epsilon$. Given $\epsilon$, the spread in $x$ is given by $\beta$, and the spread in $p_x$ is given by $\gamma$. Figure sourced from~\cite{beam-dynamics}.}
  \label{fig:LHC:emittance}
\end{figure}

Though the (normalized) emittance is an invariant, there is still flexibility to control the shape of the beam locus in phase space.
The parameter that controls the spread along the position axis is called $\beta$ and the spread along the momentum axis $\gamma$\footnote{These, along with a cross-term parameter $\alpha$, are called the \textit{Twiss parameters} in Hamiltonian mechanics. Not to be confused with the relativistic $\beta=v$ (with $c$=1) and $\gamma = \frac{1}{\sqrt{1-\beta^2}}$ - there are unfortunately too few Greek letters for all the uses physics has for them.}.
This is illustrated in Figure~\ref{fig:LHC:emittance}.
In order to maximize the luminosity, $\beta$ is desired to be as small as possible in order to reduce the size of the beam in the transverse plane\footnote{There is an emittance corresponding to each of the longitudinal direction and the two transverse directions. For the colliding beams the relevant quantity is the size in the transverse directions - the two transverse directions are treated symmetrically, assumed to have an equal $\epsilon^*$ and $\beta^*$.}, at the expense of increasing the spread in momentum space.
The minimum $\beta$ reached at the interaction point is called $\beta^*$.

As mentioned above, the rate of proton-proton interactions is roughly $\mathcal{L}\sigma_{pp} \approx 10^9~\text{s}^{-1}$, and the rate of bunch crossings, or \textit{events}\footnote{Since the bunch crossing occurs in a short span of time (much less than the time between bunch crossings), the products of all interactions in a single bunch crossing are recorded in the detector at once and so the bunch crossing is the natural discretization unit.}, is $40$ MHz ($=25$ ns between bunch crossings).
This means that there is more than one proton-proton interaction per bunch crossing, a phenomenon known as \textit{pile-up}.
The average number of interactions per bunch crossing $\mu$ is a direct measurement of the luminosity and amount of pile-up.
Also as mentioned above, almost all proton-proton interactions are ``uninteresting'', so that even with $\mu>1$ most events need to be thrown away due to disk writing and space requirements; in ATLAS there is a trigger (Section~\ref{sec:ATLAS:trigger}) that decides in real-time whether or not to store an event based on whether at least one of the interactions was ``interesting''.
In addition, even if an event is triggered, it is almost certain that there is at most one ``interesting'' interaction in the event; however, the remaining interactions cover the detector with a mostly uniform soft layer of particles, introducing a major source of noise in the energy measurements of physics objects in the detector.
In particular, jets (Chapter~\ref{ch:Jets}) can include energy deposits from pile-up interactions, requiring this contribution to be subtracted and calibrated, which is the focus of Chapters~\ref{ch:NI} and~\ref{ch:GenNI}.

In ATLAS, the luminosity is measured and monitored~\cite{LUCID2} using detectors downstream from the interaction point, almost along the beam line, and using the well-measured total differential inelastic proton-proton cross section.

The luminosity gives the rate of proton-proton collisions per unit time.
The \textit{integrated luminosity} $\int\mathcal{L}dt$ is the integral of the luminosity over the data-taking run, and has units of inverse area.
The total number of events produced over a data-taking run for a process with cross section $\sigma$ is $N = \sigma\int\mathcal{L}dt$, and so the integrated luminosity is a key quantity for understanding the statistical power of a dataset.

\section{Data Taking History and Future}
\label{sec:LHC:data}
The LHC turned on for the first time on 10 September 2008~\cite{the-lhc}, though due to ``the incident'' on 19 September 2008~\cite{Bajko:1168025}, it did not turn back on again until 20 November 2009, and did not start recording collisions at its target (less than design) center-of-mass energy of $7$ TeV until 30 March 2010.
In 2010 and 2011 the LHC ran successfully at $\sqrt{s}=7$ TeV and in 2012 this was bumped up to $\sqrt{s}=8$ TeV.
The data-taking run consisting of 2010-2012 is called \textit{Run 1}, and was particularly important as ATLAS and CMS jointly announced the discovery of the Higgs boson~\cite{HIGG-2012-27,CMS-HIG-12-028} based on this dataset.
Figure~\ref{fig:LHC:data_run1} shows the luminosity, $\mu$ distributions, and integrated luminosity over time in Run 1\footnote{What is being shown is the ATLAS recorded luminosity, which is slightly less than the LHC delivered luminosity.}.
The luminosity was gradually ramped up over the course of the Run to a maximum of almost $8\times10^{33}~\text{cm}^{-2}\text{s}^{-1}$.
This was achieved in part by increasing $\mu$ from $5-15$ in 2010 and 2011 to $10-30$ in 2012.
The total integrated luminosity delivered was $48.1~\text{pb}^{-1}$ in 2010 and $5.46~\ifb$ in 2011 at $\sqrt{s}=7~\TeV$ and $22.8~\ifb$ in 2012 at $\sqrt{s}=8~\TeV$~\cite{luminositypublic_run1}.

\begin{figure}[htbp]
  \centering 
  \subfloat[]{\includegraphics[width=1.0\textwidth]{figures/{lumivstime_2010_2012}.eps}}\\
  \subfloat[]{\includegraphics[width=0.5\textwidth]{figures/{mu_2010_2012}.eps}}
  \subfloat[]{\includegraphics[width=0.5\textwidth]{figures/{intlumivsyear_2010_2012}.eps}}
  \caption{(a) Luminosity vs time in Run 1. (b) $\mu$ distributions in 2010-2011 at $\sqrt{s}=7$ TeV and in 2012 at $\sqrt{s}=8$ TeV. (c) Integrated luminosity vs time in Run 1. Figure sourced from~\cite{luminositypublic_run1}.}
  \label{fig:LHC:data_run1}
\end{figure}

After Run 1 the LHC was shut down for \textit{Long Shutdown 1} until resuming collisions at $\sqrt{s}=13$ TeV in June 2015.
The center-of-mass energy remained at $13$ TeV for data-taking in 2015, 2016, 2017, and 2018, a period known as \textit{Run 2}.
The searches in this Thesis use data taken during Run 2 - the search in Chapter~\ref{ch:HBSM} uses data from 2015-2016, while the search in Chapter~\ref{ch:CWoLa} uses the entire Run 2 dataset 2015-2018.
Figure~\ref{fig:LHC:data_run2} shows the luminosity, $\mu$ distributions, and integrated luminosity over time in Run 2\footnote{Again, what is being shown is the ATLAS recorded luminosity, which is slightly less than the LHC delivered luminosity.}.
The luminosity was gradually ramped up over time and reached a maximum of $21\times10^{33}~\text{cm}^{-2}\text{s}^{-1}$ in 2018.
The $\mu$ distribution was higher than in Run 1, broadly ranging from $10-60$ with a mode at around $\langle \mu \rangle \approx 30$.
The total integrated luminosity delivered was $4.2~\ifb$ in 2015, $38.5~\ifb$ in 2016, $50.2~\ifb$ in 2017, and $63.3~\ifb$ in 2018, for a total of $156~\ifb$~\cite{luminositypublic}.

\begin{figure}[htbp]
  \centering 
  \subfloat[]{\includegraphics[width=0.5\textwidth]{figures/{lumivstime_2015}.pdf}}
  \subfloat[]{\includegraphics[width=0.5\textwidth]{figures/{lumivstime_2016}.pdf}}\\
  \subfloat[]{\includegraphics[width=0.5\textwidth]{figures/{lumivstime_2017}.pdf}}
  \subfloat[]{\includegraphics[width=0.5\textwidth]{figures/{lumivstime_2018}.pdf}}\\
  \subfloat[]{\includegraphics[width=0.5\textwidth]{figures/{mu_2015_2018}.pdf}}
  \subfloat[]{\includegraphics[width=0.5\textwidth]{figures/{intlumivsyear_2015_2018}.pdf}}
  \caption{(a,b,c,d) Luminosity vs time in 2015, 2016, 2017, 2018, respectively. (b) $\mu$ distributions in Run 2. (c) Integrated luminosity vs time in Runs 1 and 2. Figure sourced from~\cite{luminositypublic}.}
  \label{fig:LHC:data_run2}
\end{figure}

At time of writing, the LHC is currently in \textit{Long Shutdown 2}, which is projected to last until 2021.
Data taking in \textit{Run 3} will last from 2021 to 2024~\cite{lhc-commissioning} at the design center-of-mass energy of $14$ TeV and a peak luminosity of $2.0\times10^{34}~\text{cm}^{-2}\text{s}^{-1}$~\cite{Boyd:2020qox}, with $\langle \mu \rangle\approx 55$, similar to data-taking in 2018.
Run 3 is targeted to provide about $100~\ifb$ per year, and a total of $300-400~\ifb$ total~\cite{lhc-commissioning-run3}.

The LHC will then be shut down again for \textit{Long Shutdown 3} until 2027.
From 2027 to 2030 the LHC will be running in \textit{Run 4}, then shutting down for \textit{Long Shutdown 4} until \textit{Run 5} from 2032 to 2034~\cite{lhc-commissioning}.
Runs 4 and 5 are also known as \textit{high-luminosity LHC} or \textit{HL-LHC}~\cite{hl-lhc}, as the luminosity will be significantly increased.
The target peak luminosity is $5.0\times10^{34}~\text{cm}^{-2}\text{s}^{-1}$~\cite{Boyd:2020qox}, achieved mostly by decreasing $\beta^*$ by a factor of almost $2$, and increasing the number of protons per bunch also by a factor of almost $2$ relative to Run 2.
The pile-up in HL-LHC will be at $\langle \mu \rangle$ between $150$ and $200$, presenting a significant challenge to suppress this amount of noise, especially in low energy jets.
Due to this increased luminosity, HL-LHC is expected to deliver almost $3000~\ifb$ over the course of the two Runs.

