The Large Hadron Collider produces particle collisions at the highest energies ever observed in a scientific experiment.
This apparatus is built to test the predictions of the Standard Model of Particle Physics, including the existence and properties of the Higgs boson.
As a proton-proton collider, quarks and gluons are produced in abundance, which quickly fragment and hadronize into collimated showers of energy called jets.
These jets are detected and measured in the ATLAS detector, which is built around the point of the proton-proton collisions to observe the products of these interactions.

Three major original research efforts are presented using data from proton-proton collisions observed in ATLAS.
The first analyzes events with jets and photons to search for a beyond-the-Standard-Model decay of the Higgs boson.
The second utilizes novel techniques in weak supervision to perform a generic data-driven resonance search in events with two jets.
The third formalizes the calibration of the jet energies observed in the ATLAS detector, and further proposes a new method to improve this calibration with machine learning.

The work presented here addresses some of the key questions in particle physics today.
By searching for new physics, it is possible to shed light on the nature of the Higgs boson and the possibility of physics beyond the Standard Model. 
These searches focus on processes involving multiple jets in the final state, which motivates innovations in the reconstruction of jet energies.
In addition to setting new bounds on theoretically interesting models, the innovations in object reconstruction and analysis techniques developed in this work can be applied in other ATLAS efforts using currently available data or data gathered in the future.
