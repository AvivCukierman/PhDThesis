They say it takes a village to raise a PhD student.
While many people have contributed in some way to this milestone, my PhD experience was characterized by a few intense relationships which resulted in extraordinary research output.

The first acknowledgement goes to Ariel.
When I first started in the SLAC ATLAS group I had some vague notions about what I wanted to work on over the course of my degree.
Ariel inundated me with new creative ideas, far more than I could possibly work on, and all of them presenting exciting new avenues of discovery.
As an adviser, Ariel provided me not only with guidelines of interesting things to look into, but also with the freedom to pursue my own interests or go off on unpredicted tangential projects, for which I am extremely grateful.
These tangential projects turned into some of the most valuable original research and publications presented here.

Perhaps no one single person is more responsible for helping me attain my PhD than Ben.
As first a more senior student, and then an easily accessible postdoc, Ben provided both a mentorship and peer role which was immensely helpful for me to understand my own ideas and how they could be used and communicated in the community.
Of the seven publications I have been majorly involved with over the course of my PhD, three of them were produced by our two-person team, including three of the four key chapters in this Thesis, and an additional two were produced in small teams involving the two of us.
This represents a remarkable collaboration, and my thanks goes to Ben for being such an excellent research partner. 

The fourth key chapter resulted from another two-person team involving Francesco.
Francesco served as a close mentor and colleague during the key transition period between a student and a researcher that is the goal of a PhD.
Through him I learned what questions were the right ones to ask and how to do the work necessary to answer them satisfactorily.
My thanks goes to Francesco for providing the scaffolding for me to grow into a fully-fledged researcher.

While my relationships to the following people were not as intense as those mentioned above, they all contributed in some way and can rest assured that I would not have been as successful without them.
%Inevitably I will forget to mention someone important, but if I interacted with you in any way over the course of the last six years then you probably contributed to this work.

My thanks goes to the other senior members of the SLAC ATLAS group - Su Dong, Rainer, Charlie, Michael, Caterina, Lauren, Catrin, and Nicoletta - who provided much-needed feedback on my work from a senior perspective.
I have special gratitude to the SLAC computing team, who provided an excellent service which enabled my computationally-heavy research and allowed me to almost entirely avoid using the CERN grid over the course of my PhD.
There were the many postdocs and older students who served as more on-the-ground sources of knowledge: Max, Qi, Zihao, Rafael, Valentina, Zijun, Pascal, Katie, and Ke.
Finally, there were the younger students who helped me clarify my ideas and also reminded me to lighten up a little bit: Nicole, Murtaza, Jannicke, Sanha, and Rachel.

I am extremely grateful for my reading committee, which other than Ariel and Su Dong comprises Pat Burchat and Natalia Toro, and for my defense chair David Reis.
Their valuable feedback was necessary for communicating my ideas to a wider audience.

Some special thanks goes to Adrienne and Maria for making sure I got paid, stayed on track with my requirements, and graduated on time.

There is the entire ATLAS collaboration, including too many individuals to mention with whom I spent an inordinate amount of time speaking to remotely and in person.
These include in particular members of the JET/MET group, or anyone who attended HCW, HFSF, or BOOST, and with whom I had detailed conversations about exciting projects. 
There are also the Exotics and HDBS groups, whose conveners and members allowed us to pursue our non-standard analysis ideas and supported us through the byzantine ATLAS publication process.

Before joining ATLAS I already had some research projects under my belt, and my thanks goes to Mike, Adam, and Peter for giving me valuable research experience which set me up for success in my PhD.

All of the professors and teachers who taught me how to learn in graduate school, undergraduate, and even high school and middle school laid the foundations for me being able to discover new knowledge via research.

Finally I turn to personal relationships.
Celine has been a constant source of happiness in my life for the past decade, and I can only hope we continue to make each other happy forever.
My thanks also go out to all my other friends from Stanford, MIT, and from TJ or before who never let me down when I was looking to destress and take some distance from school.
Ari has been there my whole life and probably understands me better than anyone - I couldn't ask for a brother whose thought processes are so similar and yet different enough to my own to allow for such spirited and deep questions and discussions.
The original acknowledgements of course go to my very cool and totally normal parents, who raised me to be kind, generous, and responsible, provided support and love my entire life, and taught me to always exceed expectations and to constantly look for ways to grow and better myself.
