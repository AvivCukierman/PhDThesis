\section{Introduction}
\label{sec:SM:intro}
The \textit{Standard Model of Particle Physics}, or \textit{SM}, is the overarching theory that describes Physics at the most fundamental level.
In the SM, \textit{particles} are localizations of \textit{quantum fields}, and these particle-fields account for both matter and all known forces or interactions other than gravity.
The SM is a remarkably well-measured and well-tested theory.
Figure~\ref{fig:SM:tests} shows the cross sections of a variety of Physics processes measured with the ATLAS detector (Chapter~\ref{ch:ATLAS}) and their theoretical predictions using the SM.
Every single measurement agrees with the theoretical prediction within the uncertainties.
\begin{figure}[htbp]
  \centering 
  \subfloat[]{\includegraphics[width=0.8\textwidth]{figures/{SM_tests}.pdf}}
  \caption{Summary of several cross section measurements at ATLAS, compared to their theoretical predictions using the SM. Figure sourced from~\cite{ATL-PHYS-PUB-2019-024}.}
  \label{fig:SM:tests}
\end{figure}

The SM describes three fundamental forces, each with a corresponding gauge symmetry group.
The electromagnetic and weak forces are combined into a single \textit{electroweak} interaction under SU(2)$\times$U(1) (Section~\ref{sec:SM:EW}). 
The SU(2) symmetry is spontaneously broken by the Higgs boson, giving rise to massive bosons and to the mass of quarks.
The strong force is described by \textit{quantum chromodynamics}, with symmetry group SU(3) (Section~\ref{sec:SM:QCD}).
The strong force is unique in that it gets weaker at higher energy scales, a phenomenon called \textit{asymptotic freedom}, which cause quarks to \textit{confine} into multi-quark \textit{hadrons} like protons and neutrons.
The SM is summarized in Section~\ref{sec:SM:summary}.
Despite the successes of the SM, there are still unexplained phenomena like dark matter and dark energy and other curious features (which, if left unexplained, are at least unsatisfying) like the quark mass hierarchy (Section~\ref{sec:SM:BSM}).
There are many theories \textit{beyond the SM}, or \textit{BSM}, which attempt to answer these questions, many of which predict new particles that may be produced at the LHC (Chapter~\ref{ch:LHC}).
The two searches presented in this Thesis (Chapter~\ref{ch:HBSM} and Chapter~\ref{ch:CWoLa}) are both looking for evidence of any BSM particles.
The analysis in Chapter~\ref{ch:HBSM} is searching for BSM decays of the Higgs boson, while the analysis in Chapter~\ref{ch:CWoLa} is searching for generic new massive particles decaying hadronically.

This Chapter serves as a brief overview of the SM on a technical level; there are a variety of textbooks and other sources which provide much more detailed and complete information.
Some textbooks in particular are~\cite{Schwartz:2013pla,Peskin:1995ev,Weinberg:1995mt,griffiths_particles}, which are the textbooks the Author used to learn about quantum field theory and the SM as a student.
In addition to serving as the primary sources for most of this Chapter, these textbooks were the catalysts which (in part) spurred the interest in Particle Physics ultimately resulting in the work presented in this Thesis.

\section{Electroweak Sector}
\label{sec:SM:EW}
The Standard Model is described by \textit{Yang-Mills} or \textit{non-Abelian} gauge theories, with interactions mediated by vector bosons between fermionic matter particles with spin $\frac{1}{2}$.
The electromagnetic and weak forces are unified using the \textit{Glashow-Weinberg-Salam} model for electroweak unification, with gauge symmetry SU(2)$\times$U(1).
The gauge bosons corresponding to these symmetries are $W_\mu^a$, with $a$ running over the $3$ generators of SU(2), and $B_\mu$, respectively.
The U(1) symmetry does not correspond to the familiar electromagnetic charge, but rather corresponds to a \textit{hypercharge} $Y$; we will see that the photon corresponds to a linear combination of $B_\mu$ and one of the $W_\mu^a$.
There is in addition a complex doublet $H$ with hypercharge $\frac{1}{2}$ called the \textit{Higgs multiplet}.
Altogether, the kinetic term for these fields is
\begin{align}
  \mathcal{L}_\text{kin} &= -\frac{1}{4}(W_{\mu\nu}^a)^2 -\frac{1}{4}B_{\mu\nu}^2\nonumber\\
              &+(D_\mu H)^\dagger(D_\mu H) + m^2 H^\dagger H - \lambda (H^\dagger H)^2,\\
              D_\mu H &= \partial_\mu H - igW_\mu^a\tau^aH - \frac{1}{2}ig'B_\mu H,
\end{align}
where $B_{\mu\nu} = \partial_\mu B_\nu - \partial_\nu B_\mu$ is the field strength;
$W_{\mu\nu}^a = \partial_\mu W_\nu^a - \partial_\nu W_\mu^a + gf^{abc}W_\mu^bW_\nu^c$ is the field strength of the $W$, and $f^{abc} = \epsilon^{abc} \ne 0$ the totally anti-symmetric tensor are the structure constants of SU(2);
$m$ and $\lambda$ are free parameters of the theory;
$D_\mu$ is the covariant derivative, with corresponding couplings $g$, $g'$ which are free parameters of the theory;
$\tau^a = \frac{1}{2}\sigma^a$ are the canonically normalized generators of SU(2), with $\sigma^a$ the Pauli matrices;
and the factor of $\frac{1}{2}$ in front of $B_\mu H$ is due to the hypercharge $\frac{1}{2}$ of the $H$.

The gauge bosons are all massless, and in fact terms corresponding to the mass of these fields (e.g., $B^\mu B_\mu$) would violate gauge invariance.
These bosons (or rather, linear combinations of them) gain mass by the SU(2) symmetry being spontaneously broken by the Higgs.
The Higgs potential term, $V(H) = m^2 |H|^2 - \lambda |H|^4$, has a minimum away from $H=0$, inducing a vacuum expectation value (VEV).
Without loss of generality we let the VEV be real, $v=\frac{m}{\sqrt{\lambda}}$, and in the lower component, so that expanding about the minimum we have
\begin{align}
H = \exp\left(2i\frac{\pi^a\tau^a}{v}\right) \begin{pmatrix}0\\\frac{v}{\sqrt{2}}+\frac{h}{\sqrt{2}}\end{pmatrix},
\end{align}
where we have introduced a new scalar field $h$ which corresponds to motion about the minimum at $v$; and the $\pi^a$ are the explicit SU(2) gauges.

We choose $\pi^a=0$ to simplify the calculations.
Then expanding the covariant derivative terms, we get:
\begin{align}
  |D_\mu H|^2 &= g^2\frac{v^2}{8}\left[ (W_\mu^1)^2+(W_\mu^2)^2+\left(\frac{g'}{g}B_\mu-W_\mu^3\right)^2\right]\nonumber\\
  &+\text{terms involving $h$}
\end{align}

The terms not involving $h$ correspond to mass terms for the gauge bosons, while the terms that do involve $h$ correspond to interactions between $h$ and the gauge bosons and itself.

We identify a massless and a massive boson with
\begin{align}
Z_\mu = \cos\theta_w W_\mu^3 - \sin\theta_w B_\mu,\\
A_\mu = \sin\theta_w W_\mu^3 + \cos\theta_w B_\mu,
\end{align}
with $\tan \theta_w = \frac{g'}{g}$; $\theta_w$ is called the Weinberg angle.
$A_\mu$ corresponds to the familiar photon, with mass $0$, and $Z_\mu$ is the $Z$ boson, which has mass $m_Z = \frac{1}{2\cos\theta_w}gv$.

The coupling to $A$ and therefore the normal electromagnetic charge of the gauge bosons is determined by $g[A_\mu,W_\mu^a\tau^a] = g\sin\theta_w W^3W^a[\tau^3,\tau^a]$, with the electromagnetic coupling strength
\begin{align}
e = g\sin\theta_w = g'\cos\theta_w
\end{align}
Clearly, with $a=3$, the charge is $0$ - the $Z$ boson is neutral.
$W_1$ and $W_2$ do not have definite charges, so we define instead $\tau^\pm = \frac{1}{\sqrt{2}}\left(\tau^1\pm i\tau^2\right)$, so that $[\tau^3,\tau^\pm] = \pm \tau^\pm$, and the (linear combination of the) $W$ bosons coupling to $\tau^\pm$ have charge $\pm$, which turn out to be $W^\pm = \frac{1}{\sqrt{2}}\left(W^1 \mp iW^2\right)$.
In addition to $W\pm$ having electric charge $\pm$, their mass is $m_W = \frac{gv}{2}$.
In particular, one immediate prediction is that $m_W<m_Z$.

The Higgs boson itself is a scalar, and uncharged under either hypercharge or SU(2).
However, it does interact via $3$- and $4$-point interactions with the $W^\pm$ and $Z$ bosons; the search in Chapter~\ref{ch:HBSM} targets Higgs bosons produced via this interaction in the vector-boson-fusion mode.
There are also $3$- and $4$-point self-interactions of the Higgs.
The mass of the Higgs is $m_h = \sqrt{2}m$, which again is a free parameter of the theory.

The four original free parameters $m,\lambda,g,g'$ are therefore related to experimentally observable quantities: $\alpha_e = \frac{e^2}{4\pi} = \frac{1}{137}$ is the fine structure constant; $m_Z = 91.2~\GeV$; $m_W = 80.4~\GeV$; and $m_h = 125~\GeV$.
In particular, $\sin^2\theta_w=0.22$ and $v=246~\GeV$.

\subsection{Fermions}
\label{sec:SM:EW_matter}
The coupling of the weak interaction to fermions is \textit{chiral}, meaning the interactions are different between left- and right-hand components of the fermion fields.
In fact, the weak interaction couples only to left-handed fermions, and the SU(2) symmetry is often written as SU(2)$_L$ because of this.
There are $3$ generations each of the \textit{leptons} and the \textit{quarks}, whose left-handed components are doublets under SU(2):
\begin{align}
L^i = \begin{pmatrix}\nu_{eL}\\e_L\end{pmatrix},\begin{pmatrix}\nu_{\mu L}\\\mu_L\end{pmatrix},\begin{pmatrix}\nu_{\tau L}\\\tau_L\end{pmatrix}\\
Q^i = \begin{pmatrix}u_{L}\\d_L\end{pmatrix},\begin{pmatrix}c_{L}\\s_L\end{pmatrix},\begin{pmatrix}t_{L}\\b_L\end{pmatrix},
\end{align}
where $i$ goes over the $3$ generations; and their corresponding right-handed components, which are singlets:
\begin{align}
e^i_R = e_R,\mu_R,\tau_R\\
\nu^i_R = \nu_{eR},\nu_{\mu R},\nu_{\tau R}\\
u^i_R = u_R,c_R,t_R\\
d^i_R = d_R,s_R,b_R
\end{align}

The fields are appropriately named to correspond to the familiar leptons and quarks - the electron $e$, the muon $\mu$, and the tau particle $\tau$; the neutrinos; the up type quarks up, charm, and top; and the down type quarks down, strange, and bottom.
It should be noted that the right-handed neutrinos, which are not charged under U(1) and are singlets under SU(2), do not interact via the electroweak force; they are also color neutral (Section~\ref{sec:SM:QCD}), so they are called \textit{sterile neutrinos}.

Furthermore, the quark fields listed above are in the interaction eigenstate basis, which is different from the mass eigenstate basis, as will be described below.

The charges of these fields under hypercharge and SU(2) are given in Table~\ref{tab:SM:charges} (as well as the charge under SU(3) (Section~\ref{sec:SM:QCD}).
\begin{table}[]
\centering
\caption{Charges of Standard Model fields under U(1) hypercharge, SU(2), and SU(3). - means the field does not transform under the symmetry group, i.e. a singlet. $\Box$ means the field transforms in the fundamental representation of the group, i.e. a doublet for SU(2) and a triplet for SU(3).}
\label{tab:SM:charges}
\begin{tabular}{l c c c c c c c}
\hline
Field  & $L = \begin{pmatrix}\nu_L\\e_L\end{pmatrix}$  & $e_R$  & $\nu_R$ & $Q = \begin{pmatrix}u_L\\d_L\end{pmatrix}$ & $u_R$ & $d_R$ & $H$ \\
SU(3) & - & - & - & $\Box$ & $\Box$ & $\Box$ & - \\
SU(2) & $\Box$ & - & - & $\Box$ & - & - & $\Box$ \\
U(1)  & $-\frac{1}{2}$ & $-1$ & $0$ & $\frac{1}{6}$ & $\frac{2}{3}$ & $-\frac{1}{3}$ & $\frac{1}{2}$
\end{tabular}
\end{table}

The interaction terms between the fermions and the gauge bosons are:
\begin{align}
\mathcal{L}_\text{int} &= i\bar{L}_i\left(\cancel{\partial}-ig\cancel{W}^a\tau^a-ig'Y_L\cancel{B}\right)L_i\nonumber\\
            &+ i\bar{Q}_i\left(\cancel{\partial}-ig\cancel{W}^a\tau^a-ig'Y_Q\cancel{B}\right)Q_i\nonumber\\
              &+ i\bar{e}_R^i\left(\cancel{\partial}-ig'Y_e\cancel{B}\right)e_R^i\nonumber\\
              &+ i\bar{\nu}_R^i\left(\cancel{\partial}-ig'Y_\nu\cancel{B}\right)\nu_R^i\nonumber\\
              &+ i\bar{u}_R^i\left(\cancel{\partial}-ig'Y_u\cancel{B}\right)u_R^i\nonumber\\
              &+ i\bar{d}_R^i\left(\cancel{\partial}-ig'Y_d\cancel{B}\right)d_R^i,
\end{align}
where, e.g., $\cancel{A} = \gamma^\mu A_\mu$, with $\gamma^\mu$ the Dirac $\gamma$ matrices;
$Y$ are the hypercharges corresponding to the given set of fields (Table~\ref{tab:SM:charges});
and there are implicit projection operators $P_L = \frac{1}{2}(1-\gamma_5)$, $P_R = \frac{1}{2}(1+\gamma_5)$.

The familiar elementary charge corresponds to the coefficient of coupling of the fields with $A_\mu$.
Expanding out in terms of $W$ and $B$, we have (for the electron and neutrino fields, and the same, \textit{mutatis mutandis}, for the up and down quark fields):
\begin{align}
  \mathcal{L}_\text{int} &= \bar{e}_L^i\left(-\frac{1}{2}g\cancel{W}^3+g'Y_L\cancel{B}\right)e_L^i\nonumber\\
              &+ \bar{\nu}_L^i\left(\frac{1}{2}g\cancel{W}^3+g'Y_L\cancel{B}\right)\nu_L^i\nonumber\\
              &+ g' Y_e \bar{e}_R^i\cancel{B}e_R^i +g' Y_\nu \bar{\nu}_R^i\cancel{B}\nu_R^i\nonumber\\
              &+ \text{off-diagonal terms}
\end{align}
The off-diagonal terms contain terms like, e.g., $\bar{e_L^i}g(\cancel{W}^1-\cancel{i}W^2)\nu_L = \bar{e_L^i}g\cancel{W^+}\nu_L$.
These are the flavor-changing charged current interactions with the $W$ boson - e.g., $e$ to $\nu_e$ or $c$ to $s$.

Changing basis to the $A$ and $Z$ bosons, we then have:
\begin{align}
  \mathcal{L}_\text{int} &= e\left(\left(-\frac{1}{2}+Y_L\right)\bar{e}^i_L\cancel{A}e^i_L
              + \left(\frac{1}{2}+Y_L\right)\bar{\nu}^i_L\cancel{A}\nu^i_L
              + Y_e \bar{e}^i_R\cancel{A}e^i_R + Y_\nu \bar{\nu}^i_R\cancel{A}\nu^i_R\right)\nonumber \\
              &+ \text{$Z$ terms}
\end{align}
Just to be clear, $e=g\sin\theta_w$ is a constant.
Reading off Table~\ref{tab:SM:charges}, we therefore have that the electromagnetic charges of the left- and right-handed charged leptons $e$ are $-1$ and that the electromagnetic charges of the left- and right-handed neutrinos $\nu$ are $0$.
Examining the same terms involving the left- and right-handed $u$ and $d$ fields yields electromagnetic charges of $+\frac{2}{3}$ for the up-type quarks $u$ and $-\frac{1}{3}$ for the down-type quarks $d$.

As with the gauge bosons, mass terms like $\bar{e}_L e_R$ break the gauge invariance, and mass terms arise due to couplings with the Higgs boson.
These terms are called \textit{Yukawa couplings}, e.g. for the electron
\begin{align}
  \mathcal{L}_\text{mass} = -y_e\bar{L}He_R + \text{h.c.}
\end{align}
will generate a mass term $-m_e (\bar{e}_Le_R+\bar{e}_Re_L)$ with $m_e = \frac{y_e}{\sqrt{2}}v $ after the symmetry is spontaneously broken and the Higgs boson gets a VEV.
These terms only give mass to the charged leptons and the down-type quarks.
For the up quarks, we use a slightly different term for gauge invariance:
\begin{align}
  \mathcal{L}_\text{mass} = -Y^d_{ij}\bar{Q}^i H d_R^j - iY_{ij}^u\bar{Q}^i\sigma_2H^*u_R^j + \text{h.c.}
\end{align}
where $i,j$ run over the quark generations, and $Y^d$ and $Y^u$ are general mixing matrices between the generations.
After symmetry breaking we then have
\begin{align}
  \mathcal{L}_\text{mass} = -\frac{v}{\sqrt{2}}\left(Y_{ij}^d\bar{d}_L^id_R^j + Y_{ij}^u\bar{u}_L^iu_R^j\right) + \text{h.c.}
\end{align}
We diagonalize $Y^d$ and $Y^u$ separately,
\begin{align}
  Y^d &= U_dM_dK_d^\dagger\\
  Y^u &= U_uM_uK_u^\dagger
\end{align}
with $U$ and $K$ unitary matrices and $M$ diagonal matrices with real positive eigenvalues.
With the transformations $u_L\rightarrow U_u u_L$, $d_L\rightarrow U_d d_L$, $u_R\rightarrow K_u u_R$, $d_R\rightarrow K_d d_R$ the $U$ and $K$ terms disappear, and we have
\begin{align}
  \mathcal{L}_\text{mass} = -m_j^d \bar{d}_L^jd_R^j - -m_j^u \bar{u}_L^ju_R^j +\text{h.c.}
\end{align}
where $m_j^d$ and $m_j^u$ are the quark masses - the diagonal elements of $\frac{v}{\sqrt{s}}M^d$ and $\frac{v}{\sqrt{s}}M^u$, respectively.

In diagonalizing, the $U$ matrices come up only in the interactions with $W\pm$, via terms like
\begin{align}
  W^+_\mu\bar{u}_L^i\gamma^\mu\left(U_u^\dagger U_d\right)^{ij} d_L^j + W^-_\mu\bar{d}_L^i\gamma^\mu\left(U_d^\dagger U_u\right)^{ij} u_L^j
\end{align}
So mixing occurs between the quark generations via interactions with the $W^\pm$ bosons subject to the matrix $V=\left(U_u^\dagger U_d\right)$, known as the \textit{Cabibbo-Kobayashi-Maskawa}, or \textit{CKM} matrix.
There are $3$ free angles corresponding to mixing in $i,j$ space, and $1$ free complex phase corresponding to CP violation:
\begin{align}
V = \begin{pmatrix} 1 & 0 & 0 \\ 0 & \cos\theta_{23} & \sin\theta_{23} \\ 0 & -\sin\theta_{23} & \cos\theta_{23} \end{pmatrix}\times
\begin{pmatrix} \cos\theta_{13} & 0 & \sin\theta_{13}e^{i\delta} \\ 0 & 1 & 0 \\-\sin\theta_{13}e^{i\delta} & 0 & \cos\theta_{13} \end{pmatrix}\times
\begin{pmatrix} \cos\theta_{12} & \sin\theta_{12} & 0 \\ -\sin\theta_{12} & \cos\theta_{12} & 0 \\ 0 & 0 & 1 \end{pmatrix}
\end{align}
The empirical values are $\theta_{12}=13.04\pm0.05^\circ$, $\theta_{13}=0.201\pm0.011^\circ$, $\theta_{23}=2.38\pm0.06^\circ$, and $\delta=68.8\pm4.6^\circ$.
In particular, the CKM matrix is mostly diagonal, since all the $\theta$ angles are small; the largest is $\theta_{12}$ (mixing between the first and second generation).

The leptons also have Yukawa couplings, generating their masses.
However, the neutrinos are massless\footnote{Actually, neutrinos are known to have very small but positive masses. More on this in Section~\ref{sec:SM:BSM}.}, allowing a simultaneous diagonalization of $Y^\nu$ and $Y^e$.
This implies any mixing between the lepton generations is not allowed, and lepton number for each lepton generation is conserved under the weak interaction.

As mentioned above, while Yukawa interactions explain how lepton and quark masses are possible, the actual values are free parameters of the theory.
The experimental values are summarized in Table~\ref{tab:SM:masses}.
Both the quark masses and the charged lepton masses span many orders of magnitude, raising a potential hierarchy problem (more on this in Section~\ref{sec:SM:BSM}).

\begin{table}[]
\centering
\caption{Lepton and quark masses~\cite{PDG}.}
\label{tab:SM:masses}
\begin{tabular}{c c c c c c c c}
\hline
Quark & $d$ & $s$ & $b$ & $u$ & $c$ & $t$ & \\
Mass [MeV] & $4.7$ & $96$ & $4.18\times10^3$ & $2.2$ & $1.28\times10^3$ & $173.1\times10^3$\\
\hline
Lepton & $e$ & $\mu$ & $\tau$ & $\nu_e$ & $\nu_e$ & $\nu_\tau$ & \\
Mass [MeV] & $0.511$ & $105.7$ & $1777$ & $<\mathcal{O}(10^{-6})$ & $<\mathcal{O}(10^{-6})$ & $<\mathcal{O}(10^{-6})$
\end{tabular}
\end{table}

\section{Quantum Chromodynamics}
\label{sec:SM:QCD}
The strong force is described by a symmetry group under SU(3).
The $3$ dimensions are referred to as \textit{color charges} (red, green, and blue when names are required\footnote{Totally unrelated to colors in the visual spectrum that we are used to. However, the analogy does work a little, as a state with all three color charges is has $0$ net color charge, i.e. ``white'' or colorless.}), and so the strong force is referred to as \textit{Quantum Chromodynamics}, or \textit{QCD}.
The gauge boson associated with the theory is the gluon, and of the fermionic fields only the quarks are charged under this interaction.
The quarks (here denoted by $\psi_f$, labeled by the quark flavor $f$) transform under the fundamental representation of SU(3), meaning each of the $6$ flavors are length $3$ column vectors, or triplets.
The gluons $G_\mu^a$ transform in the adjoint representation as $3\times 3$ matrices, with $8$ color-anticolor states (the \textit{octet}) corresponding to the $8$ Gell-Mann matrices.

The Lagrangian of QCD is:
\begin{align}
  \mathcal{L}_\text{QCD} = i\bar{\psi_f}\cancel{D}_\mu\psi_f-\frac{1}{4}(G_{\mu\nu}^a)^2
\end{align}
where $\cancel{D}=\gamma^\mu D_\mu$; $D_\mu = \partial_\mu - ig_sG_\mu^aT^a$ is the covariant derivative; $T^a=\frac{\lambda^a}{2}$ are the generators of SU(3) ($\lambda^a$ are the Gell-Mann matrices); $G_{\mu\nu}^a = \partial_\mu G_\nu^a-\partial_\nu G_\mu^a + g_s f^{abc} G_\mu^b G_\nu^c$, with $f_{abc}$ the structure constants of SU(3), which are totally anti-symmetric.

We see immediately that the set of fundamental interactions are: $2$ quarks + $1$ gluon, either with a color exchange or no color exchange; $3$ gluons; and $4$ gluons.

In contrast to the electroweak theory (Section~\ref{sec:SM:EW}), there is no Higgs boson to break the symmetry, and so in some sense the model is much simpler.
However, phenomenologically and empirically QCD is much more complicated, due to the running of the coupling $g_s$.

In every theory scattering cross sections and other observables are calculated perturbatively in the interaction constant.
Formally some of these terms, which correspond to loops in Feynman diagrams, are infinite; in order to account for these infinities calculations can instead be carried out by comparing calculations between two finite scales, so that the infinities cancel, a technique called \textit{renormalization}.
One consequence of renormalization (which is necessary to make finite predictions) is that the effective strength of the coupling $g$ in the theory changes, or \textit{runs} depending on the scale being probed.
This running of the coupling is encoded in the \textit{beta function}:
\begin{align}
  \frac{\partial}{\partial \ln \mu}g = \mu\frac{\partial}{\partial\mu}g = \beta(g),
\end{align}
where $\mu$ is the energy scale being probed.

In QCD\footnote{Like everything else, $\beta$ is calculated perturbatively in $g$. Given here is the one-loop contribution.} $\beta(g) = -\frac{g^3}{(4\pi)^2}\beta_0$, with $\beta_0=(\frac{11}{3}n_\text{c} - \frac{2}{3} n_\text{f})$; $n_\text{c}=3$ is the number of colors, and $n_\text{f}=6$ is the number of flavors, so $\beta_0=7>0$.
This implies that in QCD the coupling strength gets weaker at higher energies, and stronger at lower energies, a phenomenon known as \textit{asymptotic freedom}\footnote{The couplings for the electroweak theory, $g$ and $g'$, also run but much less, and stay perturbative at all energy scales.}.
Said another way, at low energy scales $g_s$ grows so large that perturbative calculations are impossible and the theory breaks down; at high energy scales $g_s$ is small and perturbative calculations can be performed.
The running of the coupling (expressed in terms of $\alpha_s = \frac{g_s^2}{4\pi}$) can be seen in Figure~\ref{fig:SM:alphas}.
It can be seen that the coupling gets too strong and the theory becomes non-perturbative around $\mathcal{O}(\GeV)$.

\begin{figure}[htbp]
  \centering 
  \subfloat[]{\includegraphics[width=0.8\textwidth]{figures/{atlas_alphas}.pdf}}
  \caption{Strong coupling $\alpha_s$ as a function of energy scale $Q$, theoretical predictions and experimental measurements. Figure sourced from~\cite{Aaboud:2017fml}.}
  \label{fig:SM:alphas}
\end{figure}

Because of the increasing strength of the strong force at low energies, quarks and gluons are \textit{confined} into color-neutral composite particles, or \textit{hadrons}.
In order to pull apart two quarks that compose a hadron, the larger the distance between them the greater the strength of the interaction, and it becomes energy favorable to pull quark-antiquark pairs out of the vacuum to create new hadrons, a process called \textit{hadronization}.

A lone quark or gluon produced in a hard-scatter process will first \textit{fragment}, radiating gluons which in turn convert into quark-antiquark pairs, etc., and then hadronize once low enough energy scales are reached.
This collimated hadronic shower leads to \textit{jets} observed in the detector (Chapter~\ref{ch:Jets}).

Hadrons composed of $2$ quarks are called \textit{mesons}, and those composed of $3$ quarks are called \textit{baryons}; hadrons with four or more constituent quarks (\textit{tetraquarks}, \textit{pentaquarks}) have been observed~\cite{Aaij:2016iza,Aaij:2016nsc} but are very rare.
The familiar protons are baryons composed of two up quarks and one down quark.

When protons collide in the LHC (Chapter~\ref{ch:LHC}), the energy scales are high enough such that the individual constituent quarks and gluons do interact rather than the entire hadron.
The QCD factorization theorem allows the cross section calculation to be factorized between the hard-scatter and perturbative $2\rightarrow N$ process and the non-perturbative hadronic structure.
However, even with this simplification, the interaction between the constituent partons has to be averaged over the distribution of momenta each could be carrying, at the energy scale of the interaction.
This information is contained in the \textit{parton distribution function}~\cite{Amoroso:2020lgh,DelDebbio:2018siw}, or \textit{PDF}, $f_q(x,\mu^2)$, which is the probability distribution function for a parton $q$ to be carrying momentum fraction $x$ at energy scale $\mu^2$.
The hard-scatter interactions can either occur via the \textit{valence partons}, which are the constituent up and down quarks, or via \textit{sea partons}, which are virtual quarks of any flavor or gluons in the interior of the proton.
A calculation of the PDFs at two energy scales can be seen in Figure~\ref{fig:SM:PDF}.
At high $x$, the valence quarks dominate, but at lower $x$ the sea partons, especially the gluons, become more prominent.

\begin{figure}[htbp]
  \centering 
  \subfloat[]{\includegraphics[width=0.8\textwidth]{figures/{nnpdf}.png}}
  \caption{Proton PDFs at (left) $\mu^2=10~\GeV^2$ and (right) $\mu^2=10^4~\GeV^2$ calculated with NNPDF3.1 at next-to-next-to leading order (NNLO). Figure sourced from~\cite{Ball:2017nwa}.}
  \label{fig:SM:PDF}
\end{figure}

\section{Summary}
\label{sec:SM:summary}
A summary of the SM can be seen in Figure~\ref{fig:SM:SM_summary}, before and after spontaneous symmetry breaking of SU(2).
\begin{figure}[htbp]
  \centering 
  \subfloat[]{\includegraphics[width=1.0\textwidth]{figures/{SM_complete}.jpg}}\\
  \subfloat[]{\includegraphics[width=0.5\textwidth]{figures/{SM_particles}.png}}
  \subfloat[]{\includegraphics[width=0.5\textwidth]{figures/{SM_interactions}.png}}
  \caption{Summary of SM. (a) Fields before and after electroweak symmetry breaking. (b) After electroweak symmetry breaking - particle content. (b) After electroweak symmetry breaking - interactions. Figures sourced from~\cite{SM_wiki}.}
  \label{fig:SM:SM_summary}
\end{figure}

Among the spin $\frac{1}{2}$ fermions, there are $3$ generations each of quarks and leptons, split into up-type and down-type.
The $6$ quarks are the (up-type) up, charm, top, and (down-type) down, strange, bottom; the $6$ leptons are the (charged) electron, muon, and tau, and their respective neutral neutrino partners.
There are also $4$ vector gauge bosons - the photon, with mediates the electromagnetic force; the $W^\pm/Z$ bosons, which mediate the weak interaction; and the gluon, which mediates the strong force.
Finally there is the Higgs boson, which spontaneously breaks the SU(2) electroweak symmetry and acquires a vacuum expectation value, which has spin $0$.

The up-type quarks each have electric charge $+\frac{2}{3}$; the down-type quarks each have charge $-\frac{1}{3}$; the charged leptons each have charge $-1$; enabling interactions with the photon; the neutrinos each have charge $0$ and do not interact via the electromagnetic force.
Each of these fermions can interact with the $W$ and $Z$ bosons; the quarks can have flavor-changing interactions according to the CKM matrix; the leptons do not mix generations via the weak force.
Only the quarks are colored and interact with the gluon via the strong force.

The phenomenology of the strong force implies color confinement, which leads to hadrons like the proton and neutron, which are the constituents of the nuclei of atoms, and the rest of the particle zoo like pions, kaons, etc.

The Higgs boson gives mass via Yukawa couplings to the charged leptons and the quarks.
It also gives mass to the $W^\pm$ and $Z$ bosons.
The $W^\pm$ bosons are charged under electromagnetism and therefore interact with the photon.
Finally, each of the gauge bosons other than the photon has self-interactions.

\section{Beyond the Standard Model}
\label{sec:SM:BSM}
Despite the successes of the SM, there are still underexplained or unexplained natural phenomena.
``Underexplained'' phenomena refer to features within the SM that are either aesthetically displeasing (also called ``unnatural''), or where there is a discrepancy between the SM prediction and the observed value.
``Unexplained'' phenomena refer to observations of effects that have no corresponding elements in the SM outlined above.

As the SM attempts to describe Physics on the most fundamental level, any unsatisfying or missing pieces indicate that the theory is not complete, implying there must exist a theory of new Physics, or \textit{beyond the SM} (\textit{BSM}).

Some of these problems with the SM are outlined below, in (subjective) order of increasing weakness of the SM.
There are many BSM models~\cite{Lykken:2010mc,Lee:2019zbu,Ellis:2009pz,Ellis:2012zz,Virdee:2016,Halkiadakis:2014qda}, and some of these models account for some of the problems with the SM listed below.
Some of the more popular models are included with the problem they account for.

%QCD non-perturbative
As mentioned in Section~\ref{sec:SM:QCD}, below a certain energy scale QCD becomes non-perturbative, and the theory cannot make predictions.
In particular, there is no fundamental theory describing hadrons, and descriptions have to rely on phenomenological models, e.g. to describe the hadron PDFs.
While not necessarily a failure of the theory itself, this does indicate there may exist a different formulation of the SM which would be able to calculate quantities and make predictions about hadrons from a fundamental level.
One alternative formulation is lattice QCD~\cite{Gupta:1997nd}, which analyzes QCD on a lattice in spacetime, to some but limited success.

%Muon g-2
As mentioned in Section~\ref{sec:SM:intro}, the SM makes predictions that can be calculated very precisely, and these predictions have been remarkably well-tested.
There are only a couple of precision measurements that are in tension with the SM predictions, of the tens or hundreds of precision SM tests.
The muon anomalous magnetic moment~\cite{Blum:2013xva}, or muon $g-2$, has been calculated~\cite{Aoyama:2012wk} up to $10$th order, including weak and hadronic effects in addition to the primary quantum electromagnetic effects, yielding a precision of $10$ decimal places.
This quantity has also been measured~\cite{Bennett:2006fi} to an uncertainty of $0.54$ parts per million.
There is a tension of almost $3\sigma$ (with $\sigma$ equal to the experimental uncertainty) between the experimental and theoretical values, which may be an indication of new physics or may be a statistical fluctuation.
A new experiment~\cite{Chapelain:2017syu} at Fermilab will reduce the experimental uncertainty by a factor of around $4$, which can elucidate whether this is a real tension or not.

There is also evidence from LHCb~\cite{Alves:2008zz} of tension with the SM in $B$-hadron decays~\cite{Capdevila:2017bsm}, at the level of $3-4\sigma$.
This also remains to be seen whether it is a true tension indicative of new physics or a statistical fluctuation (or unaccounted-for experimental bias).

%Quark hierarchy | 
As mentioned in Section~\ref{sec:SM:EW}, the quark and charged lepton masses are allowed in the SM via Yukawa couplings with the Higgs boson.
However, there is a large \textit{hierarchy}, meaning many orders of magnitude difference in free parameters of the model.
This is especially apparent for the quarks, where the top quark mass and the up quark mass differ by a factor of $10^5$. 
This indicates that there may be some theory at higher energies that explains this observed difference~\cite{Xing:2014sja}.

%Neutrino masses | Majorana, Seesaw
%Majorana neutrinos, Seesaw mechanism
The hierarchy problem gets drastically worse when considering the neutrino masses.
As discussed in Section~\ref{sec:SM:EW}, in the vanilla SM neutrinos are massless, and cosmological constraints~\cite{Ade:2013zuv} place limits on the sum of the neutrino masses at $<1$ eV.
However, observations of neutrinos oscillating between flavors~\cite{Fukuda:1998mi} indicate that neutrinos must have finite masses, which is consistent with cosmological models~\cite{Battye:2013xqa}.
The neutrino oscillation observations measure only differences between neutrino masses of different generations, with measurements of~\cite{Araki:2004mb,Evans:2013pka,PDG} $\Delta m_{12}^2=7.5\times10^{-5}~\text{eV}^2$ and $\Delta m_{23}^2=2.5\times10^{-3}~\text{eV}^2$, where the subscripts label the neutrinos in the mass basis (which is different from the interaction basis, similar to quarks).
These are only mass differences, but the implication is that there must necessarily exist some neutrino with mass $>0.05~\text{eV}$.

The neutrino masses can be generated via Yukawa couplings in the same manner as the charged leptons and the quarks, which is called \textit{Dirac masses}.
However, there is an even larger hierarchy between the neutrinos and the lepton masses (a factor of $10^9$ between the tau and the tau neutrino), indicating these masses may come from a different mechanism~\cite{King:2003jb}.
For example, the neutrinoless double beta decay process (two neutrinos decaying to two protons with no associated missing energy corresponding to neutrinos), which has not yet been observed and would violate the SM~\cite{Schwingenheuer:2012zs}, would imply that neutrinos are their own anti-particles, allowing a \textit{Majorana mass} term in the electroweak Lagrangian.

%Strong CP problem | Axions

%Higgs boson mass | SUSY

%Matter-antimatter symmetry

%Dark matter | WIMPs
%Gravity | String theory
%Dark energy | 
