\section{Gaussian Invariance Lemma}
\label{sec:NI:lemma}

{\it Let $X\sim \mathcal{N}(\mu,\sigma)$ and $f$ be some function such that $f'(x)>0$.  Then, $f(X)\sim\mathcal{N}(\mu',\sigma')$ if and only if $f(x)$ is linear in $x$.}

\vspace{5mm}

\noindent {\bf Proof.} The converse is a well-known result, and can be obtained directly from application of Equation~\ref{eqn:NI:newdist}.

Now suppose that $f(X)\sim\mathcal{N}(\mu',\sigma')$.  Let $Y=(X-\mu)/\sigma$ and define 
\begin{align}
g(y)=\frac{f(\sigma y+\mu)-\mu'}{\sigma'},
\end{align}
so that $Y$ and $Z=g(Y)$ both have a standard normal distribution. Furthermore,
\begin{align}
g'(y) = \frac{\sigma}{\sigma'}f'(\sigma y+\mu) > 0,
\end{align}
so $g$ is monotonic.

We then can write for any $c$:
\begin{align}
\nonumber
\Phi(c)=\Pr(Y<c)&=\Pr(g(Y)<g(c))\\\nonumber
&=\Pr(Z<g(c))\\
&=\Phi(g(c)),
\end{align}
Where $\Phi(x)$ is the normal distribution cumulative distribution function. Since $\Phi$ is invertible, we then have that $g(c)=c$.  Inserting the definition of $g$ then gives us the final result:
\begin{align}
f(x)=\frac{\sigma'}{\sigma} (x-\mu)+\mu'.\hspace{1 cm}\Box
\end{align}

\section{Closure of the Mean}
\label{sec:NI:mean_nonclosure}
{\it The closure of jets reconstructed from truth jets with $E_\text{T} = x$ and $f(x)=f_{me}(x)$ is given to first order by $C\approx 1-\frac{1}{2}\frac{f''(x)}{f'(x)^3}\frac{\sigma(x)^2}{x}$.}

\vspace{5mm}

\noindent  {\bf Derivation.}
We begin by Taylor expanding $f^{-1}(y)$ about $y=f(x)$:
\begin{align}
f^{-1}(y) &= \sum_{n=0}^\infty \frac{1}{n!}\left(f^{-1}\right)^{(n)}\left(f(x)\right)\cdot\left(y-f(x)\right)^n\nonumber\\
&=\sum_{n=0}^\infty \frac{1}{n!}g_n(x)\cdot\left(y-f(x)\right)^n,
\end{align}
where $g_n(x) \equiv (f^{-1})^{(n)}(f(x))$ means the $n$th derivative of $f^{-1}(y)$, evaluated at $y=f(x)$. Plugging this into Equation~\ref{eqn:NI:closuredef}, we have
\begin{align}
C &= \frac{1}{x}\int dy \rho_{Y|X}(y|x) f^{-1}(y)\nonumber\\
&=\sum_{n=0}^\infty \frac{1}{n!}\frac{g_n(x)}{x}\int dy \rho_{Y|X}(y|x) \left(y-f(x)\right)^n\nonumber\\
&=\sum_{n=0}^\infty \frac{1}{n!}\frac{g_n(x)}{x} \mu_n(x),
\end{align}
where $\mu_n(x)$ are the standard central moments $\mu_n(x) = \mathbb{E}\left[\left(Y-\mathbb{E}\left[Y\right]\right)^n\middle| X=x\right]$, since by definition $f(x)=\mathbb{E}[Y|X=x]$.

The first few central moments are independent of the distribution $\rho_{Y|X}$.  In particular, $\mu_0 = 1$ is the normalization, and $\mu_1 = 0$. Writing these terms out, we have
\begin{align}
C =\frac{g_0(x)}{x}+\sum_{n=2}^\infty \frac{1}{n!}\frac{g_n(x)}{x} \mu_n(x).
\end{align}
Noting that $g_0(x) = f^{-1}(f(x)) = x$,
\begin{align}
C &=1+\sum_{n=2}^\infty \frac{1}{n!}\frac{g_n(x)}{x} \mu_n(x).\label{eqn:NI:closureseries}
\end{align}

\noindent We see that, if $f$ is linear, then so is $f^{-1}$, and so $g_n = 0$ for all $n\ge 2$. Then Equation~\ref{eqn:NI:closureseries} reduces to $C=1$, and numerical inversion closes, as was found in Equation~\ref{eqn:NI:closure_linear_proof}.

It will be instructive to expand out the first few terms of Equation~\ref{eqn:NI:closureseries}. We note that, by definition, $\mu_2(x) = \sigma(x)^2$ is the variance, and $\mu_3(x) = \sigma(x)^3\gamma_1$ defines the skew $\gamma_1$. Then we have
\begin{align}
C &=1+\frac{1}{2}\frac{g_2(x)}{x}\sigma(x)^2+\frac{1}{6}\frac{g_3(x)}{x}\sigma(x)^3\gamma_1+\sum_{n=4}^\infty \frac{1}{n!}\frac{g_n(x)}{x} \mu_n(x).\label{eqn:NI:closureseriesexpand}
\end{align}

Suppose we are given an arbitrary distribution specified by its moments $\mu_n(x)$. Then the requirement that closure is satisfied in the form of the right hand side of Equation~\ref{eqn:NI:closureseries} converging to $1$ exactly imposes strict constraints on the function $g(x)$, so that only for a highly specific choice of $g$ and therefore $f$ is closure achieved. Thus in general we do not expect closure to be satisfied for an arbitrary initial distribution $\rho_{Y|X}$.

We note that, since we expect the derivatives $g_n(x)$ and the moments $\mu_n(x)$ to grow considerably slower than $n!$ for functions $f$ and distributions $\rho_{Y|X}$ encountered at the LHC, we expect Equation~\ref{eqn:NI:closureseries} to converge, and Equation~\ref{eqn:NI:closureseriesexpand} gives the dominant contributions to the non-closure, i.e.
\begin{align}
C \approx 1+\frac{1}{2}\frac{g_2(x)}{x}\sigma(x)^2+\frac{1}{6}\frac{g_3(x)}{x}\sigma(x)^3\gamma_1.
\end{align}

If $\rho_{Y|X}$ is symmetric or near-symmetric, or if the third derivative of $g$ is small, such that $g_3(x)\sigma(x)\gamma_1 \ll g_2(x)$, then the dominant contribution to the non-closure is just
\begin{align}
C \approx 1+\frac{1}{2}\frac{g_2(x)}{x}\sigma(x)^2.
\end{align}

\noindent We further note that
\begin{align}
g_2(x) &= (f^{-1})^{(2)}(f(x)) = -\frac{f''(x)}{f'(x)^3}\nonumber\\
\rightarrow C &\approx 1-\frac{1}{2}\frac{f''(x)}{f'(x)^3}\frac{\sigma(x)^2}{x}.\hspace{1 cm}\Box
\label{eqn:NI:closureseriesgaussian}
\end{align}

\newpage
\section{Calibrated Resolution of the Mean}
\label{sec:NI:calibrated_resolution_calculation}
{\it The calibrated resolution of jets reconstructed from truth jets with $E_\text{T} = x$ and $f(x)=f_{me}(x)$ is given to first order by $\frac{\sigma(x)}{f'(x)}$.}

\vspace{5mm}

\noindent {\bf Derivation.}
We note that, expanding $f^{-1}(y)$ about $y=f(x)$ out to one derivative, and using the definitions of $g_n(x)$ and $\mu_n(x)$ from the previous section,
\begin{align}
(f^{-1}(y))^2 \approx g_0(x)^2+2g_0(x)g_1(x)(y-f(x))+g_1(x)^2(y-f(x))^2,
\end{align}
so that
\begin{align}
\mathbb{E}\left[Z^2\middle| X=x\right]&=\int dy \rho_{Y|X}(y|x) (f^{-1}(y))^2\nonumber\\
&\approx\int dy \rho_{Y|X}(y|x) \left(g_0(x)^2+2g_0(x)g_1(x)(y-f(x))+g_1(x)^2(y-f(x))^2\right)\nonumber\\
&=g_0(x)^2\mu_0(x)+2g_0(x)g_1(x)\mu_1(x)+g_1(x)^2\mu_2(x)\nonumber\\
&=g_0(x)^2+g_1(x)^2\sigma(x)^2.\hspace{5mm}\text{($\mu_1=0$ by construction)}
\end{align}
Out to one derivative we also have that (as derived in the previous section)
\begin{align}
\mathbb{E}\left[Z\middle| X=x\right]^2 &\approx g_0(x)^2\nonumber\\
\rightarrow \sigma\left(Z|X=x\right)^2 &= \mathbb{E}\left[Z^2\middle| X=x\right]-\mathbb{E}\left[Z\middle| X=x\right]^2\nonumber\\
&\approx g_1(x)^2\sigma(x)^2.
\end{align}
Then,
\begin{align}
g_1(x) = (f^{-1})'(f(x)) &= \frac{1}{f'(x)}\nonumber\\
\rightarrow \sigma\left(Z|X=x\right)^2 &\approx \frac{\sigma(x)^2}{f'(x)^2}\nonumber\\
\rightarrow \hat{\sigma}(x)=\sigma\left(Z|X=x\right) &\approx \frac{\sigma(x)}{f'(x)}. \hspace{1 cm} \Box \label{eqn:NI:resolution}
\end{align}

\newpage
\section{Closure of the Mode}
\label{sec:NI:calibrated_mode_calculation}
{\it The closure of jets reconstructed from truth jets with $E_\text{T} = x$ and $f(x)=f_{mo}(x)$ is given to first order by $C\approx 1+\frac{f''(x)}{f'(x)^3}\frac{\tilde{\sigma}(x)^2}{x}$.}

\vspace{5mm}

\noindent {\bf Derivation.}
As a reminder for the reader, for brevity, we will let $\rho_Y(y)=\rho_Y(y|x)$ and $\rho_Z(z)=\rho_Z(z|x)$, and let the parameter $x$ be understood.

We begin by supposing that the closure is not much different than 1, so that we can examine $\rho_Z(z)$ in the vicinity of $z=x$ to find the mode $z^*$. Expanding Equation~\ref{eqn:NI:newdist} about to second order in $(z-x)$:
\begin{align}
\rho_Z(z) &= f'(z)\rho_Y(f(z))\nonumber\\
&\approx \left[f'(x)+(z-x)f''(x)+\frac{(z-x)^2}{2}f'''(x)\right]\nonumber\\
&\times\left[\rho_Y(f(x))+(z-x)\rho_Y'(f(x))f'(x)+\frac{(z-x)^2}{2}\rho_Y''(f(x))f'(x)^2\right].
\end{align}
We note from the condition Equation~\ref{eqn:NI:modedef} that $\rho_Y'(f(x))=0$, so
\begin{align}
\rho_Z(z)&\approx \left[f'(x)+(z-x)f''(x)+\frac{(z-x)^2}{2}f'''(x)\right]\nonumber\\
&\times\left[\rho_Y(f(x))+\frac{(z-x)^2}{2}\rho_Y''(f(x))f'(x)^2\right]\nonumber\\
&\approx f'(x)\rho_Y(f(x))+(z-x)f''(x)\rho_Y(f(x))\nonumber\\
&+\frac{(z-x)^2}{2}\left[f'''(x)\rho_Y(f(x))+f'(x)^3\rho_Y''(f(x))\right],
\end{align}
so that
\begin{align}
\rho'_Z(z)&\approx f''(x)\rho_Y(f(x))+(z-x)\left[f'''(x)\rho_Y(f(x))+f'(x)^3\rho_Y''(f(x))\right].
\label{eqn:NI:drhoz}
\end{align}
Then the closure condition Equation~\ref{eqn:NI:modeclosuredef} gives
\begin{align}
\rho'_Z(z^*)&=0\nonumber\\
\rightarrow z^* &\approx x-\frac{f''(x)\rho_Y(f(x))}{f'''(x)\rho_Y(f(x))+f'(x)^3\rho_Y''(f(x))},
\end{align}
i.e. the mode of $\rho_Z(z)$ occurs at $z=z^*$.  Then the closure is
\begin{align}
C &= \frac{z^*}{x}\nonumber\\
&\approx 1-\frac{1}{x}\frac{f''(x)\rho_Y(f(x))}{f'''(x)\rho_Y(f(x))+f'(x)^3\rho_Y''(f(x))}\nonumber\\
&=1-\frac{1}{x}\frac{f''(x)\frac{\rho_Y(f(x))}{\rho_Y''(f(x))}}{f'''(x)\frac{\rho_Y(f(x))}{\rho_Y''(f(x))}+f'(x)^3}\nonumber\\
&=1+\frac{f''(x)}{f'(x)^3-\tilde{\sigma}(x)^2f'''(x)}\frac{\tilde{\sigma}(x)^2}{x}.
\label{eqn:NI:mode_closure_df3}
\end{align}

In practice we find that for typical response functions, higher derivatives of $f$ tend to vanish. A comparison between the two terms in the denominator of Equation~\ref{eqn:NI:mode_closure_df3} can be found in Figure~\ref{fig:NI:d_comp} for the toy model considered in Appendix~\ref{sec:NI:toy_model}; we find that $f'(x)^3 \gg \tilde{\sigma}(x)^2f'''(x)$. Thus, in practice we recommend the approximation
\begin{align}
C\approx 1+\frac{f''(x)}{f'(x)^3}\frac{\tilde{\sigma}(x)^2}{x}.\hspace{1 cm} \Box
\label{eqn:NI:mode_closure_simple}
\end{align}
The agreement between the actual and estimated closure in Figure~\ref{fig:NI:mode_closure} also confirms this approximation. Thus, in the body of this text Equation~\ref{eqn:NI:mode_closure_simple} is presented as the result, even though Equation~\ref{eqn:NI:mode_closure_df3} is technically more precise.
\begin{figure}[]
\begin{center}
  \includegraphics[width=0.9\textwidth]{figures/{d_comp}.pdf}
\end{center}
\caption{A comparison of derivative values using a toy model similar to conditions in ATLAS or CMS. In blue, $f'(x)^3$. In red, $\tilde{\sigma}(x)^2f'''(x)$. For details of the model, see Appendix~\ref{sec:NI:toy_model}.}
\label{fig:NI:d_comp}
\end{figure}

\newpage
\section{Resolution of the Mode}
\label{sec:NI:mode_resolution_calculation}
{\it The resolution of jets reconstructed from truth jets with $E_\text{T} = x$ and $f(x)=f_{mo}(x)$ is given to first order by $\hat{\tilde{\sigma}}(x)\approx \frac{\tilde{\sigma}(x)}{f'(x)}.$}

\vspace{5mm}

\noindent {\bf Derivation.}
From Equation~\ref{eqn:NI:drhoz} we have
\begin{align}
\rho''_Z(z)&\approx f'''(x)\rho_Y(f(x))+f'(x)^3\rho_Y''(f(x)).
\end{align}
Then the resolution is given as
\begin{align}
\hat{\tilde{\sigma}}(x)^2 &= -\frac{\rho_Z(z^*)}{\rho_Z''(z^*)}\nonumber\\
&\approx-\frac{f'(x)\rho_Y(f(x))}{f'''(x)\rho_Y(f(x))+f'(x)^3\rho_Y''(f(x))}\nonumber\\
&=\frac{f'(x)\tilde{\sigma}(x)^2}{f'(x)^3-f'''(x)\tilde{\sigma}(x)^2}.
\end{align}
Following the discussion in Appendix~\ref{sec:NI:calibrated_mode_calculation}, we simplify the denominator to get the approximation
\begin{align}
\tilde{\sigma}(x)^2 &\approx \frac{\tilde{\sigma}(x)^2}{f'(x)^2}\nonumber\\
\rightarrow \tilde{\sigma}(x) &\approx \frac{\tilde{\sigma}(x)}{f'(x)}.\hspace{1 cm} \Box
\end{align}

\section{Iterated Numerical Inversion Calculation}
\label{sec:NI:iterated}
{\it The closure $C_\text{new}(x)$ after iterating numerical inversion is not necessarily closer to 1 than the closure $C(x)$ after performing numerical inversion once.}

\vspace{5mm}

\noindent {\bf Derivation.}
We limit ourselves to the case that we are using the modes of the distributions $Y|X=x$ and $Z|X=x$ to calibrate, as in practice that is what is used at ATLAS and CMS for numerical inversion.

We use the estimation of the closure of the mode Equation~\ref{eqn:NI:closure_mode_text}:
\begin{align}
C(x) &\approx 1+\frac{f''(x)}{f'(x)^3}\frac{\tilde{\sigma}(x)^2}{x}\nonumber\\
\rightarrow |C(x)-1|&\approx \left|\frac{f''(x)}{f'(x)^3}\frac{\tilde{\sigma}(x)^2}{x}\right|.
\end{align}
We use the iterated numerical inversion response
\begin{align}
f_{\text{new}}(x) &= C(x)x\nonumber\\
&\approx x+\frac{f''(x)}{f'(x)^3}\tilde{\sigma}(x)^2\\
\rightarrow f'_{\text{new}}(x) &\approx 1-3\frac{f''(x)^2}{f'(x)^4}\tilde{\sigma}(x)^2\\
\rightarrow f''_{\text{new}}(x) &\approx 12\frac{f''(x)^3}{f'(x)^5}\tilde{\sigma}(x)^2.
\end{align}
Where we have ignored higher derivatives of $f(x)$\footnote{See, e.g., Figure~\ref{fig:NI:d_comp}.} and derivatives of $\sigma(x)$\footnote{For this specific counterexample, we are examining the case that $\sigma'(x)=0$, which is realistic for high pile-up conditions.}.  We also have the estimation of the resolution of the calibrated distribution Equation~\ref{eqn:NI:resolutionmode_text}
\begin{align}
\hat{\tilde{\sigma}}(x) \approx \frac{\tilde{\sigma}(x)}{f'(x)},
\end{align}

\noindent So that we can estimate the closure after iterating numerical inversion as
\begin{align}
C_\text{new}(x) &\approx 1+\frac{f''_\text{new}(x)}{f'_\text{new}(x)^3}\frac{\hat{\tilde{\sigma}}(x)^2}{x}\nonumber\\
&\approx 1+12\frac{f''(x)^3}{f'(x)^5}\tilde{\sigma}(x)^2\frac{\tilde{\sigma}(x)^2}{f'(x)^2}\frac{1}{x}\nonumber\\
&=1+\frac{12}{x}\frac{f''(x)^3}{f'(x)^7}\tilde{\sigma}(x)^4\\
\rightarrow |C_\text{new}(x)-1| &\approx \left|\frac{12}{x}\frac{f''(x)^3}{f'(x)^7}\tilde{\sigma}(x)^4\right|\\
\rightarrow \frac{|C_\text{new}(x)-1|}{|C(x)-1|} &\approx \frac{12f''(x)^2\tilde{\sigma}(x)^2}{f'(x)^4}.\label{eqn:NI:iterated_closure_ratio_app}
\end{align}
If the ratio in Equation~\ref{eqn:NI:iterated_closure_ratio_app} is greater than 1, then the closure gets worse after a second iteration of numerical inversion. $\hspace{1 cm} \Box$

\section{Corrected Numerical Inversion Calculation}
\label{sec:NI:corrected_numerical_inversion_calculation}
With $Y\mapsto Z_\text{corr} = g^{-1}(Y)$, we will get a corrected calibrated distribution $\rho_{Z_\text{corr}|X}(z|x)$. For brevity, let $\rho_{Z_\text{corr}}(z)=\rho_{Z_\text{corr}|X}(z|x)$, where it is understood we are examining the distributions around a particular value of $x$. We will again require that $g'(x)>0$, so that
\begin{align}
\rho_{Z_\text{corr}}(z) = g'(z)\rho_Y(g(z)).
\end{align}
The closure condition is then equivalent to the condition
\begin{align}
\rho'_{Z_\text{corr}}(x) = 0,
\end{align}
i.e., the mode of the distribution $Z_\text{corr}|X=x$ occurs at $x$.  We have that
\begin{align}
\rho_{Z_\text{corr}}'(z) = g''(z)\rho_Y(g(z))+g'(z)^2\rho'_Y(g(z)),
\end{align}
so that the closure condition requires
\begin{align}
0 &=\rho_{Z_\text{corr}}'(x)\nonumber\\
&=g''(x)\rho_Y(g(x))+g'(x)^2\rho'_Y(g(x))\nonumber\\
\rightarrow 0 &=g''(x)+g'(x)^2\frac{\rho'_Y(g(x))}{\rho_Y(g(x))}.
\end{align}
We suppose that $g(x)$ is close to $f(x)$, $g(x)=f(x)+\alpha(x)$, with $|\alpha(x)|\ll \tilde{\sigma}(x)$. Then we have directly from the supposition that the distribution $Y|X=x$ is approximately Gaussian about its mode $f(x)$ with width $\tilde{\sigma}(x)$ that
\begin{align}
\frac{\rho'_Y(g(x))}{\rho_Y(g(x))} &= -\frac{\left(g(x)-f(x)\right)}{\tilde{\sigma}(x)^2}.
\end{align}
Then, the closure condition gives
\begin{align}
0 &=g''(x)+g'(x)^2\frac{\rho'_Y(g(x))}{\rho_Y(g(x))}\nonumber\\
&=g''(x)-g'(x)^2\frac{g(x)-f(x)}{\tilde{\sigma}(x)^2}.
\end{align}

\section{Corrected Numerical Inversion Parameterization}
\label{sec:NI:corrected_numerical_inversion_parameterization}
We parameterize the corrected calibration function $g(x) = g(x;f(x);a_1,...,a_n)$. For the toy model used in this chapter, we use the parameterization
\begin{align}
g(x) = f(x)+\frac{a_1}{1+\exp(\frac{x-a_2}{a_3})}.
\label{eqn:NI:app_parameterization}
\end{align}

In the model considered here, and for the response functions at the LHC, the closure goes to $1$ for large $x$ and moves away from $1$ for small $x$, a natural result of Equation~\ref{eqn:NI:closure_mode_text}. Thus, the parameterization in Equation~\ref{eqn:NI:app_parameterization} includes a ``turn-off'' to recover $g(x)=f(x)$ at large $x$ (with $a_3>0$).

In practice, there is some smallest value $x=x'$ which is being studied, and which per the discussion in the above paragraph tends to have the largest non-closure. The value $x'=20$ GeV is used in this chapter, which is the lowest calibrated $E_\text{T}$ at current conditions at the LHC. For the corrected calibration curve shown in Figure~\ref{fig:NI:mode_closure_bigs}, the parameters $a_1,a_2,a_3$ are scanned over to minimize the non-closure at this value $x'$. For the corrected calibration curve shown in Figure~\ref{fig:NI:mode_closure_bigs}, the values $a_2=a_3=x'=20$ GeV and $a_1 = 5$ GeV were used.

\clearpage
\newpage
\section{Toy Model of the ATLAS/CMS Response Function}
\label{sec:NI:toy_model}
All the ``Proofs'' quoted in Chapter~\ref{ch:NI} are valid in general, regardless of the response function $R(x)$ and the underlying distributions $Y|X=x$ (within the assumptions outlined in Section~\ref{sec:NI:assumptions}). We also expect that the ``Derivations'', which are all approximate formulas, to apply in a wide variety of cases. In order to visualize some of the results, and verify the approximations, a particular model was needed in order to get numerical values. All figures made in this chapter were derived from a simple model of the ATLAS or CMS jet $E_\text{T}$ response function\footnote{Energies are measured with calorimeters and momenta are measured with tracking detectors.  In-situ corrections using momentum balance techniques constrain the momentum.  For small-radius QCD jets, the $E_\text{T}$ and $p_\text{T}$ are nearly identical.  Since the simulation-based correction of calorimeter jets is used here as a model, the $E_\text{T}$ is used throughout. }. After specifying $f(x)$ and the distributions $Y|X=x$, the calibrated distributions were constructed using the analytic form of the calibrated distributions Equation~\ref{eqn:NI:newdist}. Then the various moments were found numerically for the calibrated distribution at each value $x$.

The response function was guided both by physical intuition and by the intention to reasonably simulate response functions published by ATLAS~\cite{Aad:2011he} and CMS~\cite{Chatrchyan:2011ds,Khachatryan:2016kdb}. When there is only a small amount of energy already in a detector cell, the detector only reconstructs a small fraction of the energy put into it, because of noise thresholds and the non-compensating nature of the ATLAS and CMS detectors. Whereas if there is already a lot of energy in a detector cell, the detector reconstructs almost all of the energy put into it. Thus $f'(x)$ was designed  to be low at low values of $x$ and then to rise steadily to 1 at high values of $x$.  This intuition does not directly apply to jets that directly use tracking information (e.g. particle-flow jets in CMS), but for the sake of simplicity only one (calorimeter) jet definition is used for illustration.

$f'(x)$ was then integrated to get $f(x)$ and divided by $x$ to get $R(x)$. The resulting $R(x)$ function approximately corresponds to the $R=0.4$ anti-$k_t$~\cite{Cacciari:2008gp} central jet response at the EM scale available in Ref.~\cite{Aad:2011he} (e.g. Fig. 4a). The shapes of $f'(x)$ and $R(x)$ in this model can be seen in Figure~\ref{fig:NI:model}.

\begin{figure}[h!]
\begin{center}
  \includegraphics[width=0.9\textwidth]{figures/{model}.pdf}
\end{center}
\caption{The toy model used in this chapter to simulate conditions in ATLAS or CMS. The left plot shows $f'(x)$ and the right plot shows $R(x)$.}
\label{fig:NI:model}
\end{figure}

In this simplified model, the distributions $Y|X=x\sim\mathcal{N}(f(x),\sigma(x))$ were used. In ATLAS and CMS, $Y|X=x$ is approximately Gaussian. The constant value of $\sigma(x)=7$ GeV was used, corresponding to a calibrated resolution (Fig.~\ref{fig:NI:mean_resolution}) of about 50\% at $E_\text{T}=20$ GeV.  This is consistent with e.g.~\cite{Aad:2015ina} and has the property that $\sigma'(x) = 0$, which should be the case if pile-up is the dominant contributor to the resolution of low $E_\text{T}$ jets.
