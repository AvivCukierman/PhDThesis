Over the past century or so, a remarkable picture has emerged about the nature of the universe - that fundamentally, everything we observe around us, and every way in which those things interact, is governed by elementary \textit{particles}, which are themselves local perturbations in quantum \textit{fields}.
The theory that describes these particles and their interactions, called the \textit{Standard Model}, makes many predictions which have been confirmed to be accurate by extremely precise measurements.
Some examples of these predictions include the description of \textit{Quantum Chromodynamics}\cite{ref:TODO}, or \textit{QCD}, including the phenomenon of \textit{jets}\cite{ref:TODO} which are observed in particle detectors; and the existence of the \textit{Higgs boson}~\cite{Englert:1964et,Higgs:1964pj,Higgs:1964ia,Guralnik:1964eu}, the discovery of which~\cite{HIGG-2012-27,CMS-HIG-12-028} earned a Nobel Prize in Physics in 2012\cite{ref:TODO}.
However, there are still big unanswered questions about certain phenomena we observe around us, including the nature of \textit{Dark Matter}~\cite{Trimble:1987ee} and \textit{Dark Energy}\cite{ref:TODO}, the so-called \textit{hierarchy problem}~\cite{Nilles:1982dy}, the strong CP problem~\cite{Peccei:1977hh}, and the possible \textit{fine-tuning}\cite{ref:TODO} that allows our universe to exist.
The success of the Standard Model in describing most of the phenomena we observe, and in predicting the outcomes of experiments, lends confidence to the idea that these unexplained phenomena may be explained by particles and fields.
Some proposed new theories include \textit{Supersymmetry}~\cite{ref:TODO,Dobrescu:2000yn,Ellwanger:2003jt,Dermisek:2005ar,Chang:2008cw,Morrissey:2008gm}, \textit{Higgs multiplet} models\cite{ref:TODO}, and other more exotic theories~\cite{ref:TODO}, including exotic decays of the Higgs boson~\cite{Curtin:2013fra}. %Refs for SUSY are only for NMSSM
These theories are tested using the \textit{ATLAS} experiment\cite{PERF-2007-01} to detect the results of particle collisions in the \textit{Large Hadron Collider}, or \textit{LHC}, located at the \textit{Center for European Nuclear Research}, or \textit{CERN}.

This Thesis presents original research in efforts to search for new physics beyond the Standard Model using jets observed in the ATLAS detector, including work on improving the reconstruction of jet energies intended to improve the sensitivity of these searches.
The work presented here addresses some of the key questions in particle physics today.
By searching for new physics, it is possible to shed light on the nature of the Higgs boson and the possibility of physics beyond the Standard Model. 
These searches focus on processes involving multiple jets in the final state, which motivates innovations in the reconstruction of jet energies.
In addition to setting new bounds on theoretically interesting models,
the innovations in object reconstruction and analysis techniques developed in this Thesis can be applied in other ATLAS efforts using currently available data or data gathered in the future.

Chapter~\ref{ch:SM} gives an overview of the Standard Model and goes into detail about some relevant aspects.
In Chapter~\ref{ch:LHC}, the LHC is described, and in Chapter~\ref{ch:ATLAS}, the ATLAS experiment is detailed, including the reconstruction of jets.

The remaining chapters cover original work by the author using data gathered at the ATLAS experiment.

Two chapters are devoted to independent searches for new physics using jets at ATLAS.
Chapter~\ref{ch:HBSM} covers a search for a beyond-the-Standard-Model decay of a Higgs boson.
In Chapter~\ref{ch:CWoLa}, a generic data-driven resonance search using machine learning is described.

Two following chapters discuss work to improve the reconstruction of the energy of jets from the measurements in the ATLAS calorimeter and tracking systems.
Chapter~\ref{ch:NI} describes efforts to make rigorous the calibration of jets observed in the ATLAS detector, and Chapter~\ref{ch:Calib} discusses a new method to further improve this calibration with machine learning.

Finally, Chapter~\ref{ch:Conclusion} recounts the described efforts, puts them into context, and discusses future developments.

A few Appendices add to the body of work presented in this Thesis.
Appendix ~\ref{ch:NI_app} contains some additional studies and proofs relevant to Chapter~\ref{ch:NI}.
In Appendix~\ref{ch:Voronoi}, a novel method is proposed for subtracting the contribution of multiple simultaneous or out-of-time proton-proton collisions, or \textit{pile-up}, to the jet energy.
In Appendix~\ref{ch:Beamspot}, a new technique is studied for calculating the precise location of the proton-proton interactions in ATLAS, or \textit{beamspot}, using Bayesian inference.

\section{Units}
The unit of energy to be used most often is the $\GeV$, or giga-electronvolt, which is a billion times the energy gained by an electron when moving across a one volt potential difference.
Natural units are used, with $c=1$, so that the $\GeV$ is also the unit of measurement for mass.

\section{Coordinates}
As described in Chapter~\ref{ch:ATLAS}, ATLAS uses a right-handed coordinate system with its origin at the nominal interaction point (IP)
in the center of the detector and the \(z\)-axis along the beam pipe.
The \(x\)-axis points from the IP to the center of the LHC ring,
and the \(y\)-axis points upwards.
Cylindrical coordinates \((r,\phi)\) are used in the transverse plane,
\(\phi\) being the azimuthal angle around the \(z\)-axis.
The \textit{pseudorapidity} is defined in terms of the polar angle \(\theta\) as \(\eta = -\ln \tan(\theta/2)\).
The \textit{rapidity} is defined for an object with observed energy $E$ and momentum $p$ as \(y = \frac{1}{2}\ln\frac{E+p_Z}{E-p_Z}\); in the limit that the object's mass $m$ goes to zero, the rapidity and pseudorapidity are equal.
Angular distance is measured in units of \(\Delta R \equiv \sqrt{(\Delta\eta)^{2} + (\Delta\phi)^{2}}\).

%\section{Statistics}
