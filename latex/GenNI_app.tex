\section{Correcting for Auxiliary Variables}
\label{sec:GenNI_app:auxiliary}

In the following, note that the description of the jet energy correction applies both to the currently used method of correction in the ATLAS GSC (Section~\ref{sec:GenNI:jetreco}) for a single variable and to generalized numerical inversion (Section~\ref{sec:GenNI:genni}).
The difference between the two methods is only in the details of the derivation of $f_\theta(x)$; the former does a binned fit and is only used with a single variable at a time, while the latter does an unbinned fit and allows for fitting multiple variables at once.

Following the notation of Chapter~\ref{ch:NI}, let $X$ be a random variable representing $p_\text{T}^\text{true}$ and $Y$ be a random variable representing $p_\text{T}^\text{reco}$.
We also let $\theta$ represent some auxiliary variable on which the jet energy depends.
For demonstrative purposes suppose this effect is linear,
\begin{align}
  f_\theta(x) = \alpha(x)\theta+\beta(x).
\end{align}
We note that, using the law of total expectation,
\begin{align}
  f(x) &= \mathbb{E}_\theta\left[f_\theta(x)\right]\\
       &= \alpha(x)\mu_\theta(x)+\beta(x)
\end{align}
where $\mu_\theta(x) \equiv \mathbb{E}\left[\theta|X=x\right]$.
Here we use the shorthand $\mathbb{E}_\theta[\cdot] = \mathbb{E}[\cdot|X=x]$ to indicate the expectation taken over $\theta$ while keeping $X$ fixed.

Assuming that after the inclusive correction the calibration closes overall ($f(x)=x$) further imposes the form
\begin{align}
  f_\theta(x) = \alpha(x)(\theta-\mu_\theta(x))+x.
\end{align}

We now turn to the resolution of the jets before and after the correction.
We define
\begin{align}
  \sigma(x)^2 &\equiv \mathbb{E}\left[\left(Y-f(x)\right)^2|X=x\right],\\
  \sigma(x,\theta)^2 &\equiv \mathbb{E}\left[\left(Y-f_\theta(x)\right)^2|X=x,\theta\right],\\
  \sigma_\theta(x)^2 &\equiv \mathbb{E}\left[\left(\theta-\mu_\theta(x)\right)^2|X=x\right].
\end{align}
I.e., $\sigma(x)^2$ is the overall variance of $Y$ given $X=x$, which is simply the (square of the) resolution after the inclusive calibration but before the correction for the auxiliary variable; $\sigma(x,z)^2$ is the variance of $Y$ given both $X=x$ and $\theta$; and $\sigma_\theta(x)$ is the variance of $\theta$ itself given $X=x$.

Before the correction,
\begin{align}
  \sigma(x)^2 &=\mathbb{E}\left[Y^2|X=x\right]-f(x)^2\\
              &=\mathbb{E}_\theta\left[\mathbb{E}[Y^2|X=x,\theta]\right]-x^2\\
              &=\mathbb{E}_\theta\left[\sigma(x,\theta)^2+f(x,\theta)^2\right]-x^2\\
              &=\mathbb{E}_\theta\left[\sigma(x,\theta)^2\right]+\mathbb{E}_\theta\left[\alpha(x)^2(\theta-\mu_\theta(x)^2\right]\\
              &=\mathbb{E}_\theta\left[\sigma(x,\theta)^2\right]+\alpha(x)^2\sigma_\theta(x)^2
\end{align}
The last term, including the variance of $\theta$, comes from the fact that we have a dependence on the external variable.
This is exactly the term that gives rise to the intuition that there is an increase in the spread of the jet energy due to the dependence on $\theta$ and the spread of $\theta$ itself, and is the impetus for correcting this dependence.

For deriving the resolution after the correction, it will be important to understand the derivative of $f_\theta(x)$:
\begin{align}
  f_\theta'(x) = 1+\alpha'(x)(\theta-\mu_\theta(x)) - \alpha(x)\mu_\theta'(x).
\end{align}
The correction is $Y \mapsto \hat{Y} = f_\theta^{-1}(Y)$, so that after the correction:
\begin{align}
  \hat{f}_\theta(x) &\equiv \mathbb{E}\left[\hat{Y}|X=x,\theta\right]\\
  &\approx f_\theta(f_\theta^{-1}(x)) = x,\\
  \hat{\sigma}(x,\theta)^2 &\equiv \mathbb{E}\left[\left(\hat{Y}-\hat{f}_\theta(x)\right)^2|X=x,\theta\right]\\
  &\approx\frac{\sigma(x,\theta)^2}{f_\theta'(x)^2},
\end{align}
where both approximations come from Chapter~\ref{ch:NI} (Equation~\ref{eqn:NI:closureseries_text} and Equation~\ref{eqn:NI:resolutionseries_text}).

The response overall has closure:
\begin{align}
  \hat{f}(x) &\equiv \mathbb{E}\left[\hat{Y}|X=x\right]\\
  &= \mathbb{E}_\theta\left[ \hat{f}_\theta(x)\right]\\
  &= x
\end{align}
And the resolution overall is
\begin{align}
  \hat{\sigma}(x)^2 &\equiv \mathbb{E}\left[(\hat{Y}-\hat{f}(x))^2|X=x\right]\\
  &=\mathbb{E}_\theta\left[\mathbb{E}\left[(\hat{Y}-\hat{f}_\theta(x))^2|X=x,\theta\right]\right]\\
  &=\mathbb{E}_\theta\left[\hat{\sigma}(x,\theta)^2\right]\\
  &=\mathbb{E}_\theta\left[\frac{\sigma(x,\theta)^2}{f'(x,\theta)^2}\right]\\
  &=\mathbb{E}_\theta\left[\frac{\sigma(x,\theta)^2}{\left(1+\alpha'(x)(\theta-\mu_\theta(x)) -\alpha(x)\mu'_\theta(x)\right)^2}\right]
\end{align}

The resolution without the correction is always worse than $\mathbb{E}_\theta\left[\sigma(x,\theta)^2\right]$, due to the spread of the response due to $\theta$, and this is the value we should compare the corrected resolution to.

To gain intuition, we examine a few simple cases. $\alpha'(x)$ captures whether the dependence of the jet energy on $\theta$ changes with $x$, and $\mu'_\theta(x)$ captures whether $\theta$ itself is correlated with $X$ rather than just the fluctuations in $Y$ around $X$.

The first thing to note is that if both $\alpha'(x)=0$ and $\mu'_\theta(x)=0$, then the resolution is exactly $\mathbb{E}_\theta\left[\sigma(x,z)^2\right]$ and so strictly gets better.
However, in this case, numerical inversion is unnecessary, as $f_\theta(x) = \alpha(\theta-\mu_\theta)+x$ and so a simple correction $Y \mapsto Y-\alpha(\theta-\mu_\theta)$ is sufficient to remove this effect.
In effect this is what is done in the pile-up subtraction stage of the jet calibration (Section~\ref{sec:ATLAS:jet_calibration}), which is a simple offset and does not use numerical inversion.

Suppose then that $\mu_\theta'(x)=0$ and $\alpha'(x) \ne 0$.
If the spread of $\alpha'(x)(\theta-\mu_\theta)$ is small, then $\mathbb{E}_\theta\left[\frac{1}{(1+\alpha'(x)(\theta-\mu_\theta))^2}\right] \approx \mathbb{E}_\theta\left[1-2\alpha'(x)(\theta-\mu_\theta)\right] = 1$ and the resolution is exactly $\mathbb{E}_\theta\left[\sigma(x,z)^2\right]$; so the resolution after correction is always better.
However, the resolution suffers depending on the support of $\theta$.
If there are even a few values of $\theta$ for which $\alpha'(x)(\theta-\mu_\theta)$ is near $-1$, then the derivative goes to $0$ at those values and those values are calibrated to plus or minus infinity.
Because of this, the calibrated resolution blows up, even as the uncalibrated resolution is finite.
Even when using the trimmed Gaussian width instead of the standard deviation, if there is enough support of $\theta$ at these extreme values then the resolution can blow up.

As another case, suppose that $\alpha'(x)=0$ but $\mu_\theta'(x) \ne 0$.
Then the resolution depends on the sign of $\alpha\mu_\theta'(x)$.
Taking $\alpha>0$ for concreteness, this could either be because $\theta$ gives information about the fluctuations in $Y-X$, in which case $\mu_\theta'(x)<0$; or it could be because $\theta$ contains information about $X$ directly, in which case $\mu_\theta'(x) > 0$.
In the former case, $\alpha\mu_\theta'(x)<0$, and correcting for the dependence on $\theta$ makes the resolution better, because these fluctuations are being accounted for.
In the latter case, $\alpha\mu_\theta'(x)>0$, and correcting for the dependence on $\theta$ makes the resolution worse, because information about the correlation between $Y$ and $X$ is actually being removed by the correction.
The story is the same, \textit{mutatis mutandis}, with $\alpha<0$.

As an extreme example of the degradation of the resolution if $\alpha\mu_\theta'(x)>0$, suppose $\mu_\theta(x)=x$, so $\mu_\theta'(x)=1$, and $\alpha=1$.
Then $f_\theta(x) = (\theta-x)+x = \theta$.
So after correcting for the dependence of $Y$ on $\theta$, all the correlation between $Y$ and $X$ has been removed, and the resolution of $Y$ given $X$ is therefore infinite.
In fact, in this case it makes more sense to simply use $\theta$ as the measurement of the jet energy and calibrate that with respect to $X$, an idea which is beyond the scope of this Thesis.
